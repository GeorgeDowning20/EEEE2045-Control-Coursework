\documentclass[12pt]{article}
\input{inc/Setup/inlcude.tex}
\input{inc/Setup/Conf.tex}
\newcommand{\ex}{ex} 
\newcommand{\appendixamble}[1]{This graph was automatically generated from code using matlab (Appendix~\ref{appendix:#1})}


\newenvironment{conditions}[1][where:]
  {#1 \begin{tabular}[t]{>{$}l<{$} @{${}={}$} l}}
  {\end{tabular}\\[\belowdisplayskip]}

\begin{document}

%%%%%%%%%%%%%%%%%%%%%%%%%%%%%%%%%%%%%%%%%%%%%%%%%%%%%%%%%%%%%%%%%%%%%%%%%%%%%%%%%%%%%%%%%

\begin{titlepage}
    \centering
    \vspace*{0.5 cm}
    \includegraphics[scale = 0.4]{uon.png}\\[1.0 cm]	% University Logo
    \textsc{\LARGE University of Nottingham}\\[2.0 cm]	% University Name
    \textsc{\Large EEEE2045:}\\[0.5 cm]				% Course Code
    \textsc{\large Control Coursework}\\[0.5 cm]				% Course Name
    \rule{\linewidth}{0.2 mm} \\[0.4 cm]
    { \huge \bfseries \thetitle}\\
    \rule{\linewidth}{0.2 mm} \\[1.5 cm]

    \begin{minipage}{0.4\textwidth}
        \begin{flushleft} \large
            \emph{Author:}\\
            \theauthor
        \end{flushleft}
    \end{minipage}~
    \begin{minipage}{0.4\textwidth}
        \begin{flushright} \large
            \emph{Student Number:} \\
            20273662									% Your Student Number
        \end{flushright}
    \end{minipage}\\[2 cm]

    {\large \thedate}\\[2 cm]

    \vfill

\end{titlepage}

\pagebreak
%%%%%%%%%%%%%%%%%%%%%%%%%%%%%%%%%%%%%%%%%%%%%%%%%%%%%%%%%%%%%%%%%%%%%%%%%%%%%%%%%%%%%%%%%

\tableofcontents


\pagebreak
%%%%%%%%%%%%%%%%%%%%%%%%%%%%%%%%%%%%%%%%%%%%%%%%%%%%%%%%%%%%%%%%%%%%%%%%%%%%%%%%%%%%%%%%%

\section{Aim of the Lab}\label{sec:aim}

The aim of this Lab is to complete a set of exercises to demonstrate the understanding of the concepts of control systems. The exercises are based on the material covered in the lectures and the textbook. \\

Two design questions will also be completed, proving the understanding of the concepts of control systems, and the ability to apply them to a real world problem.\\

Matlab will be used to complete the exercises and design questions. The ability to use Matlab will be demonstrated by the through the use of commands and functions required.
The code used to complete the exercises and design questions will be included in the report.\\

Problems will be approached both mathmatically and graphically where appropriate. The graphs will be included in the report. \\

Analysis of system/plant/controller/feedback transfers functions for open and closed loop control systems will be demonstrated including the use of step response plots and root locus plots.\\

The concept of reducing transfer functions by analyzing the dominant roots is looked at and presented through the use of S-plane and Root locus plots.\\

%%%%%%%%%%%%%%%%%%%%%%%%%%%%%%%%%%%%%%%%%%%%%%%%%%%%%%%%%%%%%%%%%%%%%%%%%%%%%%%%%%%%%%%%%

\section{Approach}\label{sec:approach}

Firstly a project folder was created, to manage the workspace for the report and the matlab scripts. The necessary files were then added to the project folder including a title page and LaTex template. The files were then added to the git repository. The git repository was then pushed to github.\\

The report was writen in LaTex using TeX Live 2019/Debian. The --shell-escape flag must be used when compiling the report to give the latex script access to shell commands. This allows the latex script to run matlab's scripts from the terminal and input tikz pictures correctly .\\ 

The project required the use of matlab's control toolbox and matlab2tikz. The control toolbox and matlab2tikz were installed using the matlab package manager.\\

The matlab scripts were then written to produce the graphs required for the report. The matlab scripts were then integrated into the latex document using the matlab2tikz package.\\

Three primary matlab commands were used to create the graphs. step,rlocus and bode were used to create step response, root locus s-plane plots and bode diagrams respectively.the function matlab2tikz was then used to create a .tex file, to be included in the document.\\

The workspace as as above allows for any changes in the matlabs code, to be reflected in the report at run time, including automatically updating the graphs and appendix.\\

All figures, graphs, equations and tables were numbered, referenced and hyperlinked in the report, for easy viewing. A contents page was automatically generated, including sections,subsections, appendix and references.\\

For each exercise, the relevant mathematical notation for each system was written out in the report. The instructions were accurately followed to produce the necessary output. The graphs were included in the report using matlab2tikz. The Matlab code was then linked to the appendix. further comments and explanations were then added including graphs, equations and table were appropriate. \\

For each Question, the Initial constants were calculated from the given information in the question. Where the second question required simplification, justification and explanation was given. The problems were first solved mathematically then checked using a bode plot to ensure the correct answer was obtained graphically. The final answer was then compared against the initial specification using a step response plot of the closed loop transfer function. finally a bode plot was used to visualize the system and ascertain its bandwidth  \\






\pagebreak
%%%%%%%%%%%%%%%%%%%%%%%%%%%%%%%%%%%%%%%%%%%%%%%%%%%%%%%%%%%%%%%%%%%%%%%%%%%%%%%%%%%%%%%%%
\section{Results and Discussion}\label{sec:results}

\subsection{Exercise 1}\label{sec:ex1}
\renewcommand{\ex}{ex1} 

\begin{equation}
    G_{p}\left(s\right)=\frac{a}{s+20}
    \label{eq:\ex}
\end{equation}

\begin{equation}
    \lim_{s\to0}\left(G_{p}\left(s\right)\right)
    \label{eq:\ex.1}
\end{equation}

\begin{figure}[ht!]
    \centering
    \input{matlab/Output/\ex.tex}
    \caption{Graph showing step response of plant \eqref{eq:\ex}. \appendixamble{\ex}}
    \label{fig:\ex}
\end{figure}\FloatBarrier

The plant transfer function for a simple first order system is given by \eqref{eq:\ex} where $a$ is the gain of the system and $s$ is the Laplace variable. Steady state is achieved when s approaches 0 as described by \eqref{eq:\ex.1}. Therefore the steady state gain of the system is $a/20$.

When $a=10$ the steady state gain is $\frac{10}{20}$ which is $0.5$.

Unity gain is achieved when the gain is equal to 1 at steady state. Hence the gain of the system is equal to 1 when $a=20$. The step response of such system is shown in Figure~\ref{fig:\ex}.

The time constant is the time taken for the system to reach $1-e^{-1}$ or approximately 63.2\% of its final value. The time constant is equal to the reciprocal of the denominator at steady state. Hence the time constant is equal to 0.05 at all values of $a$.





\subsection{Exercise 2}\label{sec:ex2}

\begin{equation}
    G_{p}\left(s\right)=\frac{a}{s+a}
    \label{eq:ex2}
\end{equation}

\begin{figure}[ht!]
    \centering
    \input{matlab/Ex2.tex}
    \caption{Graph showing step response of the transfer function shown in \eqref{eq:ex2} where a = 1 (blue), 3 (red), 12 (green) and 20 (black) }
    \label{fig:ex2}
\end{figure}\FloatBarrier

\begin{table}[ht!]
    \centering
    \begin{tabular}{|c|c|}
        \hline
        a{[}arb{]} & Time Constant{[}s{]} \\ \hline
        1          & 1                    \\ \hline
        3          & 0.333                \\ \hline
        12         & 0.0883               \\ \hline
        20         & 0.05                 \\ \hline
    \end{tabular}
    \caption{Table to show the time constant of the system for different values of a}
    \label{tab:ex2}
\end{table}\FloatBarrier

Figure~\ref{fig:ex2}. shows the step response of the system for different values of $a$. The steady state gain of the system is $a/a$ which is equal to 1 for all values of $a$. The time constant of the system is shown in Table~\ref{tab:ex2}. The time constant is equal to the reciprocal of the denominator at steady state. Hence the time constant is equal to $1/a$ at all values of $a$.\\

\begin{figure}[ht!]
    \centering
    % This file was created by matlab2tikz.
%
%The latest updates can be retrieved from
%  http://www.mathworks.com/matlabcentral/fileexchange/22022-matlab2tikz-matlab2tikz
%where you can also make suggestions and rate matlab2tikz.
%
\definecolor{mycolor1}{rgb}{0.00000,0.44700,0.74100}%
%
\begin{tikzpicture}

\begin{axis}[%
width=4.442in,
height=3.393in,
at={(0.893in,0.486in)},
scale only axis,
separate axis lines,
every outer x axis line/.append style={white!40!black},
every x tick label/.append style={font=\color{white!40!black}},
every x tick/.append style={white!40!black},
xmin=0,
xmax=1,
every outer y axis line/.append style={white!40!black},
every y tick label/.append style={font=\color{white!40!black}},
every y tick/.append style={white!40!black},
ymin=0,
ymax=500000,
axis background/.style={fill=white}
]
\addplot [color=mycolor1, forget plot]
  table[row sep=crcr]{%
0	0\\
0.00460517016472295	0.0964781961520202\\
0.00921034038765356	0.202264434599783\\
0.0138155105523765	0.318256738537457\\
0.0184206807170995	0.445439770759549\\
0.0230258509400301	0.584893192455638\\
0.027631021104753	0.737800828763284\\
0.0322361913276836	0.905460717971437\\
0.0368413614924066	1.08929613087093\\
0.0414465316571295	1.29086765274405\\
0.0460517018800601	1.51188643148635\\
0.0506568720447831	1.75422870332841\\
0.055262042209506	2.01995172043098\\
0.0598672124324366	2.31131121481303\\
0.0644723825971596	2.63078054768266\\
0.0690775527618825	2.98107170552248\\
0.0736827229848132	3.36515832238365\\
0.0782878931495361	3.78630092320964\\
0.0828930633724667	4.24807460251031\\
0.0874982335371897	4.75439937337069\\
0.0921034037019126	5.30957344482886\\
0.0967085739248432	5.91830970917363\\
0.101313744089566	6.58577575028175\\
0.105918914254289	7.31763771100668\\
0.11052408447722	8.1201083935448\\
0.115129254641943	9\\
0.119734424806666	9.96478196140379\\
0.124339595029596	11.0226443461725\\
0.128944765194319	12.1825673855492\\
0.13354993541725	13.4543977074791\\
0.138155105581973	14.8489319246146\\
0.142760275746696	16.3780082875164\\
0.147365445969626	18.0546071796562\\
0.151970616134349	19.8929613085347\\
0.156575786299072	21.9086765276734\\
0.161180956522003	24.1188643150963\\
0.165786126686726	26.5422870334005\\
0.170391296909656	29.1995172040188\\
0.174996467074379	32.1131121482467\\
0.179601637239102	35.3078054770012\\
0.184206807462033	38.8107170553412\\
0.188811977626756	42.6515832240111\\
0.193417147791479	46.863009232271\\
0.198022318014409	51.4807460249867\\
0.202627488179132	56.5439937337069\\
0.207232658343855	62.0957344479975\\
0.211837828566786	68.1830970919109\\
0.216442998731509	74.857757502934\\
0.221048168954439	82.1763771102414\\
0.225653339119162	90.2010839355644\\
0.230258509283885	99\\
0.234863679506816	108.647819614329\\
0.239468849671539	119.226443461725\\
0.244074019836262	130.825673855608\\
0.248679190059192	143.543977074558\\
0.253284360223915	157.489319246088\\
0.257889530388638	172.780082874931\\
0.262494700611569	189.546071796329\\
0.267099870776292	207.929613085405\\
0.271705040999223	228.086765276734\\
0.276310211163945	250.188643150963\\
0.280915381328668	274.422870333772\\
0.285520551551599	300.995172040188\\
0.290125721716322	330.131121482584\\
0.294730891881045	362.078054770071\\
0.299336062103976	397.10717055347\\
0.303941232268699	435.515832240111\\
0.308546402433421	477.630092322594\\
0.313151572656352	523.807460249751\\
0.317756742821075	574.439937337127\\
0.322361913044006	629.95734448015\\
0.326967083208729	690.830970918876\\
0.331572253373452	757.577575029107\\
0.336177423596382	830.763771102589\\
0.340782593761105	911.010839355818\\
0.345387763925828	998.999999999942\\
0.349992934148759	1095.47819614311\\
0.354598104313482	1201.2644346173\\
0.359203274478205	1317.25673855632\\
0.363808444701135	1444.43977074581\\
0.368413614865858	1583.89319246099\\
0.373018785088789	1736.8008287492\\
0.377623955253512	1904.46071796305\\
0.382229125418235	2088.29613085382\\
0.386834295641165	2289.86765276757\\
0.391439465805888	2510.88643150934\\
0.396044635970611	2753.2287033379\\
0.400649806193542	3018.9517204017\\
0.405254976358265	3310.3112148256\\
0.409860146581195	3629.78054770065\\
0.414465316745918	3980.07170553459\\
0.419070486910641	4364.15832240123\\
0.423675657133572	4785.30092322588\\
0.428280827298295	5247.07460249716\\
0.432885997463018	5753.39937337098\\
0.437491167685948	6308.57344480127\\
0.442096337850671	6917.30970918864\\
0.446701508015394	7584.77575029101\\
0.451306678238325	8316.63771102577\\
0.455911848403048	9119.10839355807\\
0.460517018625978	9998.99999999889\\
0.465122188790701	10963.7819614306\\
0.469727358955424	12021.6443461727\\
0.474332529178355	13181.5673855626\\
0.478937699343078	14453.3977074576\\
0.483542869507801	15847.9319246093\\
0.488148039730731	17377.0082874917\\
0.492753209895454	19053.6071796302\\
0.497358380060177	20891.9613085379\\
0.501963550283108	22907.676527675\\
0.506568720447831	25117.8643150927\\
0.511173890670761	27541.2870333783\\
0.515779060835484	30198.5172040164\\
0.520384231000207	33112.1121482549\\
0.524989401223138	36306.8054770055\\
0.529594571387861	39809.7170553446\\
0.534199741552584	43650.5832240109\\
0.538804911775514	47862.0092322576\\
0.543410081940237	52479.7460249703\\
0.54801525210496	57542.993733708\\
0.552620422327891	63094.7344480109\\
0.557225592492614	69182.0970918843\\
0.561830762715545	75856.757502908\\
0.566435932880268	83175.3771102556\\
0.57104110304499	91200.0839355783\\
0.575646273267921	99998.999999986\\
0.580251443432644	109646.819614303\\
0.584856613597367	120225.443461724\\
0.589461783820298	131824.673855622\\
0.594066953985021	144542.977074572\\
0.598672124149743	158488.319246088\\
0.603277294372674	173779.082874912\\
0.607882464537397	190545.071796297\\
0.612487634760328	208928.613085373\\
0.617092804925051	229085.765276743\\
0.621697975089774	251187.64315092\\
0.626303145312704	275421.870333775\\
0.630908315477427	301994.172040155\\
0.63551348564215	331130.12148254\\
0.640118655865081	363077.054770045\\
0.644723826029804	398106.170553435\\
0.649328996252734	436514.832240097\\
0.653934166417457	478629.092322562\\
0.656236751528922	501186.233627192\\
};
\end{axis}

\begin{axis}[%
width=4.568in,
height=3.603in,
at={(0.766in,0.486in)},
scale only axis,
xmin=0,
xmax=1,
xtick={\empty},
xlabel={Time (seconds)},
ymin=0,
ymax=1,
ytick={\empty},
ylabel={Amplitude},
axis line style={draw=none},
ticks=none,
title style={font=\bfseries},
title={Step Response},
axis x line*=bottom,
axis y line*=left
]
\end{axis}
\end{tikzpicture}%
    \caption{graph showing the step response of the transfer function shown in \eqref{eq:ex2_2} }
    \label{fig:ex2_2}
\end{figure}\FloatBarrier

\begin{equation}
    G_{p}\left(s\right)=\frac{a}{s-a}
    \label{eq:ex2_2}
\end{equation}

Figure~\ref{fig:ex2_2}. shows the step response of the transfer function shown in \eqref{eq:ex2_2}. where the plant has a positive pole the system is unstable and will not converge to a steady state. the rate of change of the system is increasing and will continue to increase until the system is destroyed. 


\subsection{Exercise 3}\label{sec:ex3}
\renewcommand{\ex}{ex3} 

\begin{figure}[ht!]
    \centering
    \input{matlab/Output/\ex.tex}
    \caption{Graph showing step response of the transfer function shown in \eqref{eq:ex2} where a = 1 (blue), 3 (red), 12 (green) and 20 (black) }
    \label{fig:\ex}
\end{figure}\FloatBarrier
\subsection{Exercise 4a}\label{sec:ex4a}
\renewcommand{\ex}{ex4a}

\begin{equation}
	G_{p}\left(s\right)=\frac{10}{s^2 + as + 10}
	\label{eq:\ex}
\end{equation}


\begin{figure}[ht!]
	\centering
	\input{matlab/Output/\ex.tex}
	\caption{Graph showing step response of the transfer function shown in \eqref{eq:\ex} where a = 0.2 (red), 0.6 (green), 1.2 (blue), 1.6 (black) and 3.0 (magenta). \appendixamble{\ex}}
	\label{fig:\ex}
\end{figure}\FloatBarrier


\begin{table}[ht!]
    \centering
    \begin{tabular}{|c|c|c|}
    \hline
    a{[}arb{]} & Percentage Peak Overshoot $M_p[\%]$ & damping factor $\xi[arb]$ \\ \hline
    0.2        & 91                                  & 0.0316                    \\ \hline
    0.6        & 74                                  & 0.9487                    \\ \hline
    1.2        & 54                                  & 0.1897                    \\ \hline
    1.6        & 44                                  & 0.2530                    \\ \hline
    3.0        & 18                                  & 0.4743                    \\ \hline
    \end{tabular}
    \caption{Table to show the percentage peak overshoot and damping factor of the system \eqref{eq:\ex} for different values of a}\label{table:\ex}
    \end{table}\FloatBarrier

    \begin{equation}
    \begin{split}
        \omega_0 &= \sqrt{10}\\
        \therefore \xi &= \frac{a}{2\omega_0}\\
    \end{split}\label{eq:12}
    \end{equation}

Eq.~\eqref{eq:12} shows the working to calculate the damping factor $\xi$ in table \ref{table:\ex}. The damping factor is the ratio of the damping coefficient to the natural frequency of the system. The damping coefficient is the coefficient of the second order term in the differential equation. The damping coefficient is equal to $a$ in this case. The natural frequency is the square root of the coefficient of the first order term in the differential equation. The coefficient of the first order term  $\xi$ is equal to 10 in this case. Therefore the damping factor is equal to $\frac{a}{2\sqrt{10}}$.

Figure~\ref{fig:\ex}. shows the step response of the system for different values of $a$. The steady state gain of the system is $10/10$ which is equal to 1 for all values of $a$. The percentage peak overshoot and damping factor of the system is shown in Table~\ref{table:\ex}. when a is small the percentage peak overshoot approaches 1 and the damping factor is small. When a is large, the percentage peak overshoot is small and the damping factor is large. \\

The variable a changes the roots of the denominator of the transfer function, separating the roots as a gets larger. As the ratio of imaginary to real parts of the roots increases, the damping factor increases. As the ratio of imaginary to real parts of the roots decreases, the damping factor decreases. 

\begin{equation}
    \frac{\pi}{\omega_0\sqrt{1-\xi^2}}
    \label{eq:1}
\end{equation}

While the damping factor is less than 1, the system is underdamped. This means the system will oscillate about the steady state value. The time to maximum overshoot can only be observed when the system is underdamped and is characterized by the equation \eqref{eq:1}. when the damping factor is small, the time to maximum overshoot approaches a constant value. When the damping value approaches 1, the time to maximum overshoot approaches infinity.




\subsection{Exercise 4b}\label{sec:ex4b}
\renewcommand{\ex}{ex4b} 

\begin{equation}
	G_{p}\left(s\right)=\frac{100a}{\left(s^2 + 12s + 100\right)\left(s+a\right)}
	\label{eq:\ex}
\end{equation}

\begin{figure}[ht!]
    \centering
    \input{matlab/Output/\ex.tex}
    \caption{Graph showing step response of the transfer function shown in \eqref{eq:\ex} where a = 5 (red), 7 (green), 10 (blue), 25 (black) and 35 (magenta). \appendixamble{\ex}}
    \label{fig:\ex}
\end{figure}\FloatBarrier

\begin{figure}[ht!]
    \centering
    \input{matlab/Output/\ex_1.tex}
    \caption{Graph showing Pole-Zero Map of the transfer function shown in \eqref{eq:\ex} where a = 5 (red), 7 (green), 10 (blue), 25 (black) and 35 (magenta), The poles at -6$\pm j8$ exist for all values of a. \appendixamble{\ex}}
    \label{fig:\ex_2}
\end{figure}\FloatBarrier

Figure~\ref{fig:\ex}. shows the step response of the system \eqref{eq:\ex} for different values of $a$. The steady state gain of the system is $100a/100a$ which is equal to 1 for all values of $a$. The poles of the system are shown in Figure~\ref{fig:\ex_2}. The poles at -6$\pm j8$ exist for all values of $a$. The poles at -a and -6$\pm j8$ are considered dominant for small and large values of $a$ respectively. \\

When a is 35, the poles at -6$\pm j8$ are considered almost completely dominant and therefore the graph looks like an underdamped third order system. When a is 5, the pole at -a is slightly more dominant over the poles at -6$\pm j8$ therefore the graph looks like a critically damped second order system. The extent to which the systems order can be approximated, can be visualized with the help of the pole-zero map shown in Figure~\ref{fig:\ex_2}. The ratio between real parts of each pole determines how dominant it is, where the smaller the ratio the more dominant the pole is. 

\subsection{Exercise 5}\label{sec:ex5}
\renewcommand{\ex}{ex5}

\begin{equation}
	G_{p}\left(s\right)=\frac{10}{s + 10}
	\label{eq:\ex}
\end{equation}

\begin{equation}
	\begin{split}
		G\left(s\right) &= G_{p}\left(s\right)\cdot G_{c}\left(s\right)\\
		&=\frac{k \cdot 10}{s + 10}\\
	\end{split}
	\label{eq:\ex_1}
\end{equation}

\begin{conditions}
	G_{p}\left(s\right) & $\frac{10}{s + 10}$ \\
	G_{c}\left(s\right) & k
\end{conditions}\\

\begin{equation}
	\begin{split}
		G\left(s\right) &= G_{p}\left(s\right)\cdot G_{c}\left(s\right)\\
		&=\frac{10}{s + 10} \cdot \frac{k\left(s+a\right)}{s} \\
		&=\frac{10k\left(s+205\right)}{s\left(s+10\right)}\\
	\end{split}
	\label{eq:\ex_2}
\end{equation}

\begin{conditions}
	G_{p}\left(s\right) & $\frac{10}{s + 10}$ \\
	G_{c}\left(s\right) & k\\
	a & 205
\end{conditions}\\

\begin{figure}[ht!]
	\centering
	\input{matlab/Output/\ex.tex}
	\caption{Root Locus plot of the forward path transfer function \eqref{eq:\ex_2} of a proportional controller applied to plant \eqref{eq:\ex} \appendixamble{\ex}}
	\label{fig:\ex}
\end{figure}\FloatBarrier

Figure~\ref{fig:\ex} only has one pole on the real axis, which is at $s = -10$. This means that the system is stable. The root locus plot shows that the system is stable for all positive values of $k$. However it does not pass through out design point at $-200 \pm j200$ and does not meet the design specification. \\

Therefore in order to meet the design specification we need a second order controller for example a proportional integral controller.


\begin{figure}[ht!]
	\centering
	\input{matlab/Output/\ex_1.tex}
	\caption{Root Locus plot of the forward path transfer function \eqref{eq:\ex_2} of a proportional Integral controller applied to plant \eqref{eq:\ex} \appendixamble{\ex}}
	\label{fig:\ex_1}
\end{figure}\FloatBarrier

Figure~\ref{fig:\ex_1} shows that the system is stable for all positive values of $k$ and passes through the design point at $-200 \pm j200$, where k = 39.1. This means that the system meets the design specification and is the same value for k as mathematically calculated.

\begin{figure}[ht!]
	\centering
	\input{matlab/Output/\ex_2.tex}
	\caption{Step Response of the forward path transfer function \eqref{eq:\ex_2} of a proportional Integral controller applied to plant \eqref{eq:\ex} \appendixamble{\ex}}
	\label{fig:\ex_2}
\end{figure}\FloatBarrier

\begin{equation}
	\begin{split}
		\omega_{0} &= \frac{-\ln(\frac{\%}{100})}{t\cdot\xi}\\
		&=8.69338\ rad^s{-1}\\
	\end{split}\label{eq:\ex_2}
\end{equation}

\begin{conditions}
	\% & 2    \\
	t & 0.75s \\
	\xi & 0.6 \\
\end{conditions}\\

Figure~\ref{fig:\ex_2} shows the step response of the system. The system is stable and the output is as expected. at 20ms the amplitude is 0.948. this is outside of the specified response of $\pm 2\%$ at 20ms. At 27.3ms the response is within the $\pm 2\%$. This is because the equation \eqref{eq:\ex_2} used to calculate $\omega_0$ is based off a first order decay. The system is second order therefore is only approximated by this equation.
\pagebreak
%%%%%%%%%%%%%%%%%%%%%%%%%%%%%%%%%%%%%%%%%%%%%%%%%%%%%%%%%%%%%%%%%%%%%%%%%%%%%%%%%%%%%%%%%

\section{Design Questions and Solutions}\label{sec:Q4}
\subsection{1}
\begin{equation}
\begin{split}
    0&=s^{2}+2\omega_{0}\xi s\ +\omega_{0}^{2}    \\
    \therefore s&=-13.5\pm j\frac{3\sqrt{19}}{2}
\end{split}\label{eq:Q4_1_1}
\end{equation}

\begin{conditions}
    \omega_0 & 15 \\
    \xi & 0.9 \\
\end{conditions}\\

Equation~\eqref{eq:Q4_1_1} shows the calculation for the approximate position of the dominant poles based on the values of $\omega_0$ and $\xi$.

\begin{equation}
    \begin{split}
        G_{c}\left(s\right)=\frac{k\left(s+a\right)}{s} \hspace{1cm} G_{p}\left(s\right)=\frac{6}{15s+1}
    \end{split}\label{eq:Q4_1_2}
\end{equation}

\begin{conditions}
    k & Constant of proportionality \\
    a & Constant of integration \\
\end{conditions}\\

\begin{equation}
    \begin{split}
        0&=1\ +G_{p}\left(s\right)\cdot G_{c}\left(s\right)\\
        &=s^{2}+2\omega_{0}\xi s\ +\omega_{0}^{2}\\
        &=s^{2}+\frac{\left(6k+1\right)}{15}s+\frac{6ka}{15} \\
    \end{split}\label{eq:Q4_1_3}
\end{equation}

The plant and general model for a PI controller are given by Eq.~\eqref{eq:Q4_1_2}. Setting the denominator of the system to zero, we re-arrange the controller to fit the general equation Eq.~\eqref{eq:Q4_1_3}.

\begin{equation}
    \begin{split}
        2\omega_{0}\xi &= \frac{\left(6k+1\right)}{15} \\
        \therefore k &= 67.33\\
    \end{split}\label{eq:Q4_1_4}
\end{equation}

\begin{conditions}
    \omega_0 & 15 \\
    \xi & 0.9 \\
\end{conditions}\\

\begin{equation}
    \begin{split}
        \omega_{0}^{2} &= \frac{6ka}{15}\\
        \therefore a &= 8.354 \\
    \end{split}\label{eq:Q4_1_5}
\end{equation}

\begin{conditions}
    \omega_0 & 15 \\
    k & 67.33 \\
\end{conditions}\\

Equation~\eqref{eq:Q4_1_4}-\eqref{eq:Q4_1_5} shows the calculation to find the constants of proportionality and integration $k$ and $a$ by comparing coefficients using Eq.~\eqref{eq:Q4_1_3} such that the system meets the specification set out by $\omega_0$ and $\xi$.

\begin{equation}
    \begin{split}
        G\left(s\right)= \frac{6}{15s+1}\cdot\frac{K\left(s+8.354\right)}{s}\\
    \end{split}\label{eq:Q4_1_6}
\end{equation}

\renewcommand{\ex}{Q4_1} 

\begin{figure}[ht!]
    \centering
    \input{matlab/Output/\ex.tex}
    \caption{Root Locus plot of the forward path transfer function \eqref{eq:Q4_1_6} \appendixamble{\ex}}
    \label{fig:\ex}
\end{figure}\FloatBarrier

Figure~\ref{fig:\ex} shows the root locus plot passing through the poles at $-13.5\pm j\frac{3\sqrt{19}}{2}$ as found in Eq.~\eqref{eq:Q4_1_1}. The gain at  $-13.5\pm j\frac{3\sqrt{19}}{2}$ is 67.33 confirming the value found in Eq.~\eqref{eq:Q4_1_4} to be correct.

\begin{equation}
    \begin{split}
        G_c\left(s\right)=\frac{67.33\left(s+8.354\right)}{s}\\
    \end{split}\label{eq:Q4_1_7}
\end{equation}

\begin{figure}[ht!]
    \centering
    \input{matlab/Output/\ex_1.tex}
    \caption{Step Response of the closed loop transfer function where the plant is given by Eq.~\eqref{eq:Q4_1_2} and controller Eq.~\eqref{eq:Q4_1_7}. \appendixamble{\ex}}
    \label{fig:\ex_1}
\end{figure}\FloatBarrier

Figure~\ref{fig:\ex_1} shows the step response of the closed loop transfer function. The step response is shown to be within the specification of $\omega_0 = 15$ and $\xi = 0.9$.\\

The final solution to this question for the design of a PI controller is given by Eq.~\eqref{eq:Q4_1_7}.

\begin{figure}[ht!]
    \centering
    \input{matlab/Output/\ex_2.tex}
    \input{matlab/Output/\ex_3.tex}
    \caption{Bode Diagram of the closed loop transfer function where the plant is given by Eq.~\eqref{eq:Q4_1_2} and controller Eq.~\eqref{eq:Q4_1_7}. \appendixamble{\ex}}
    \label{fig:\ex_2}
\end{figure}\FloatBarrier

\begin{figure}[ht!]
    \centering
    \inputminted[firstline=58,lastline=58]{matlab}{matlab/Q4_1.m}
    \caption{MATLAB code used to calculate the bandwidth of the closed loop system}\label{fig:Q4code1}
\end{figure}\FloatBarrier

Figure~\ref{fig:\ex_2} shows the Bode Diagram for the closed loop transfer function. The bandwidth of the closed loop system is defined as when the gain is above -3dB. The bandwidth is calculated on matlabs using figure~\ref{fig:Q4code1}. The result is 34.785Hz.\label{sec:Q4_1}
\subsection{2}
\begin{equation}
	\begin{split}
		\frac{\theta\left(s\right)}{V_{motor}\left(s\right)} &= \frac{15}{\left(s\ +\ 150\right)\left(s\ +\ 8\right)}\\
		&= \frac{15/150}{\left(s\ +\ 8\right)}\\
		&= \frac{0.1}{\left(s\ +\ 8\right)}
	\end{split}\label{eq:Q4_2_1}
\end{equation}

Equation~\eqref{eq:Q4_2_1} shows how the transfer function for the motor can be simplified from a third order equation to a second order equation. The root at -150 is several times larger than -8, therefore the root at -8 can be considered dominant and the transfer function can be simplified to a second order equation.

\begin{equation}
	\begin{split}
		\omega_{0} &= \frac{-\ln(\frac{\%}{100})}{t\cdot\xi}\\
		&=8.69338\ rads^{-1}   \\
	\end{split}\label{eq:Q4_2_2}
\end{equation}

\begin{conditions}
	\% & 2 \\
	t & 0.75s \\
	\xi & 0.6\\
\end{conditions}\\

Equation~\eqref{eq:Q4_2_2} shows the calculation to find the natural frequency of the system. The natural frequency is found by using the formula for the natural frequency of a second order system with damping ratio $\xi$ and time constant $t$.

\begin{equation}
	\begin{split}
		0&=s^{2}+2\omega_{0}\xi s\ +\omega_{0}^{2}\\
		\therefore s&=-5.216028\pm j6.954704\\
	\end{split}\label{eq:Q4_2_3}
\end{equation}

\begin{conditions}
	\omega_0 & 8.69338 \\
	\xi & 0.6 \\
\end{conditions}\\

Equation~\eqref{eq:Q4_2_3} shows the calculation for the approximate position of the dominant poles based on the values of $\omega_0$ and $\xi$.

\begin{equation}
	\begin{split}
		G_{c}\left(s\right)=\frac{k\left(s+a\right)}{s} \hspace{1cm} G_{p}\left(s\right)=\frac{0.1}{s+8}
	\end{split}\label{eq:Q4_2_4}
\end{equation}

\begin{equation}
	\begin{split}
		0&=1\ +G_{p}\left(s\right)\cdot G_{c}\left(s\right)\\
		&=s^{2}+2\omega_{0}\xi s\ +\omega_{0}^{2}\\
		&=s^{2\ }+\left(8+0.1k\right)s+0.1ka\\
	\end{split}\label{eq:Q4_2_5}
\end{equation}

The plant and general model for a PI controller are given by Eq.~\eqref{eq:Q4_2_4}. Setting the denominator of the system to zero, we re-arrange the controller to fit the general equation Eq.~\eqref{eq:Q4_2_5}.

\begin{equation}
	\begin{split}
		2\omega_{0}\xi &= 8+0.1k\\
		\therefore k &= 24.32
	\end{split}\label{eq:Q4_2_6}
\end{equation}

\begin{conditions}
    \omega_0 & 8.6934 \\
    \xi & 0.6         \\
\end{conditions}\\

\begin{equation}
	\begin{split}
		\omega_{0}^{2} &= 0.1ka\\
		\therefore a &= 31.0752
	\end{split}\label{eq:Q4_2_7}
\end{equation}

\begin{conditions}
    \omega_0 & 8.6934 \\
    k & 24.32         \\
\end{conditions}\\
Equation~\eqref{eq:Q4_2_6}-\eqref{eq:Q4_2_7} shows the calculation to find the constants of proportionality and integration $k$ and $a$ by comparing coefficients using Eq.~\eqref{eq:Q4_2_5} such that the system meets the specification set out by $\omega_0$ and $\xi$.

\begin{equation}
	\begin{split}
		G_{p}\left(s\right)&=\frac{15}{\left(s\ +\ 150\right)\left(s\ +\ 8\right)}\\
		&= \frac{15}{s^{2}+158s+1200}\\
	\end{split}\label{eq:Q4_2_8}
\end{equation}

\begin{equation}
	\begin{split}
		G_{c}\left(s\right)&=\frac{k\left(s+a\right)}{s}\\
		&=\frac{24.32\left(s+31.075\right)}{s}\\
		&=\frac{24.32s+755.7}{s}\\
	\end{split}\label{eq:Q4_2_9}
\end{equation}

\begin{conditions}
    k & 24.32  \\
    a & 31.075 \\
\end{conditions}\\

\begin{equation}
	\begin{split}
		G\left(s\right)&=G_{p}\left(s\right) \cdot G_{c}\left(s\right)\\
		&=\frac{364.8+11340}{s^{3}+158s^{2}+1200s}\\
	\end{split}\label{eq:Q4_2_10}
\end{equation}

\begin{conditions}
    G_{p}\left(s\right) & $\left(\frac{15}{s^{2}+158s+1200}\right)$ \\
    G_{c}\left(s\right) & $\left(\frac{24.32s+755.7}{s}\right)$     \\
\end{conditions}\\

\begin{equation}
	\begin{split}
		H\left(s\right)&= \frac{G_{p}\left(s\right) \cdot G_{c}\left(s\right)}{1 + G_{p}\left(s\right) \cdot G_{c}\left(s\right)}\\
		&=\frac{364.8s+11340}{s^{3}+158s^{2}+1565+11340}\\
	\end{split}\label{eq:Q4_2_11}
\end{equation}

\begin{conditions}
    G_{p}\left(s\right) & $\left(\frac{15}{s^{2}+158s+1200}\right)$ \\
    G_{c}\left(s\right) & $\left(\frac{24.32s+755.7}{s}\right)$     \\
\end{conditions}\\

Equations~\eqref{eq:Q4_2_8}-\eqref{eq:Q4_2_11} shows the plant, controller, forward path and closed loops transfer functions as designed for the motor system.

\renewcommand{\ex}{Q4_2}

\begin{figure}[ht!]
	\centering
	\input{matlab/Output/\ex.tex}
	\caption{Root Locus plot of the closed loop transfer function \eqref{eq:Q4_2_11}. \appendixamble{\ex}}
	\label{fig:\ex}
\end{figure}\FloatBarrier

Figure~\ref{fig:\ex} shows the dominant poles passing through $-5.03 \pm 7.16$ which is close to the value of $-5.216028\pm j6.954704$ calculated in Eq.~\eqref{eq:Q4_2_3}. Therefore the system will behave as designed. The calculations simplifies the third order equation to a second order equation, therefore is only an approximation and we would not expect the roots of the actual system to be exactly the same as the calculated values.


\begin{figure}[ht!]
	\centering
	\input{matlab/Output/\ex_1.tex}
	\caption{Step Response of the closed loop transfer function \eqref{eq:Q4_2_11}. \appendixamble{\ex}}
	\label{fig:\ex_1}
\end{figure}\FloatBarrier

Figure~\ref{fig:\ex_1} shows the amplitude at 750ms to be 0.993, therefore is within the specification of $2 \%$ by 750ms.
\label{sec:Q4_2}
\pagebreak
%%%%%%%%%%%%%%%%%%%%%%%%%%%%%%%%%%%%%%%%%%%%%%%%%%%%%%%%%%%%%%%%%%%%%%%%%%%%%%%%%%%%%%%%%

\section{Summary and Conclusions}\label{sec:summary}


\pagebreak
%%%%%%%%%%%%%%%%%%%%%%%%%%%%%%%%%%%%%%%%%%%%%%%%%%%%%%%%%%%%%%%%%%%%%%%%%%%%%%%%%%%%%%%%%

\addcontentsline{toc}{section}{References}
\bibliographystyle{IEEEtranN}
\bibliography{biblist}

\pagebreak
%%%%%%%%%%%%%%%%%%%%%%%%%%%%%%%%%%%%%%%%%%%%%%%%%%%%%%%%%%%%%%%%%%%%%%%%%%%%%%%%%%%%%%%%%

\appendix
\section*{Appendices}\addcontentsline{toc}{section}{Appendices}\label{appendix:main}
\renewcommand{\thesubsection}{\Alph{subsection}}

\subsection{Matlab Code}
\subsubsection{Ex1: (\ref{sec:ex1})}\label{appendix:ex1}
\inputminted{matlab}{matlab/ex1.m}

\subsubsection{Ex2: (\ref{sec:ex2})}\label{appendix:ex2}
\inputminted{matlab}{matlab/ex2.m}

\subsubsection{Ex3: (\ref{sec:ex3})}\label{appendix:ex3}
\inputminted{matlab}{matlab/ex3.m}

\subsubsection{Ex4a: (\ref{sec:ex4a})}\label{appendix:ex4a}
\inputminted{matlab}{matlab/ex4a.m}

\subsubsection{Ex4b: (\ref{sec:ex4b})}\label{appendix:ex4b}
\inputminted{matlab}{matlab/ex4b.m}

\subsubsection{Ex5: (\ref{sec:ex5})}\label{appendix:ex5}
\inputminted{matlab}{matlab/ex5.m}

\subsubsection{Q4.1: (\ref{sec:Q4_1})}\label{appendix:Q4_1}
\inputminted{matlab}{matlab/Q4_1.m}

\subsubsection{Q4.2: (\ref{sec:Q4_2})}\label{appendix:Q4_2}
\inputminted{matlab}{matlab/Q4_2.m}
%%%%%%%%%%%%%%%%%%%%%%%%%%%%%%%%%%%%%%%%%%%%%%%%%%%%%%%%%%%%%%%%%%%%%%%%%%%%%%%%%%%%%%%%%

\end{document}