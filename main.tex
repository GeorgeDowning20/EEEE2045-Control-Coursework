\documentclass[12pt]{article}
\usepackage[colorlinks=true, allcolors=blue]{hyperref}
\usepackage[english]{babel}
\usepackage[numbers]{natbib}
\usepackage{url}
\usepackage[utf8x]{inputenc}
\usepackage{amsmath}
\usepackage{graphicx}
\usepackage{parskip}
\usepackage{fancyhdr}
\usepackage{vmargin}
\usepackage{tikz}
\usepackage{pgfplots} 
\usepackage{minted}
\usepackage{placeins}
\usepackage{tikz}
\usepackage[siunitx,european]{circuitikz}
\usepackage{amssymb}

\title{Control Coursework}								
\author{George Downing}								                        
\date{October 24, 2022}					

\graphicspath{{images/}}
\setmarginsrb{3 cm}{2.5 cm}{3 cm}{2.5 cm}{1 cm}{1.5 cm}{1 cm}{1.5 cm}

\usetikzlibrary{angles,quotes} 
\usetikzlibrary{arrows.meta}
\usetikzlibrary{calc}
\ctikzset{tripoles/mos style/arrows} 						            

\makeatletter
\let\thetitle\@title
\let\theauthor\@author
\let\thedate\@date
\makeatother

\pagestyle{fancy}
\fancyhf{}
\rhead{\theauthor}
\lhead{\thetitle}
\cfoot{\thepage}


\newcommand{\ex}{ex} 
\newcommand{\appendixamble}[1]{This graph was automatically generated from code using matlab (Appendix~\ref{appendix:#1})}


\newenvironment{conditions}[1][where:]
  {#1 \begin{tabular}[t]{>{$}l<{$} @{${}={}$} l}}
  {\end{tabular}\\[\belowdisplayskip]}

\begin{document}

%%%%%%%%%%%%%%%%%%%%%%%%%%%%%%%%%%%%%%%%%%%%%%%%%%%%%%%%%%%%%%%%%%%%%%%%%%%%%%%%%%%%%%%%%

\begin{titlepage}
    \centering
    \vspace*{0.5 cm}
    \includegraphics[scale = 0.4]{uon.png}\\[1.0 cm]	% University Logo

    
    \textsc{\Large Department of Electrical and Electronic Engineering
    Faculty of Engineering}\\[2.0 cm]	% University Name
    \textsc{\large Electrical Energy Conditioning and Control}\\[0.5 cm]				% Course Code
    \textsc{\large (EEEE2045 UNUK) (FYR1 22-23)}\\[0.5 cm]				% Course Name


    \rule{\linewidth}{0.2 mm} \\[0.4 cm]
    { \huge \bfseries \thetitle}\\
    \rule{\linewidth}{0.2 mm} \\[1.5 cm]

    \begin{minipage}{0.4\textwidth}
        \begin{flushleft} \large
            \emph{Author:}\\
            \theauthor
        \end{flushleft}
    \end{minipage}~
    \begin{minipage}{0.4\textwidth}
        \begin{flushright} \large
            \emph{Student Number:} \\
            20273662									% Your Student Number
        \end{flushright}
    \end{minipage}\\[2 cm]

    {\large \thedate}\\[2 cm]

    \vfill

\end{titlepage}

\pagebreak
%%%%%%%%%%%%%%%%%%%%%%%%%%%%%%%%%%%%%%%%%%%%%%%%%%%%%%%%%%%%%%%%%%%%%%%%%%%%%%%%%%%%%%%%%

\tableofcontents


\pagebreak
%%%%%%%%%%%%%%%%%%%%%%%%%%%%%%%%%%%%%%%%%%%%%%%%%%%%%%%%%%%%%%%%%%%%%%%%%%%%%%%%%%%%%%%%%

\section{Aim of the Lab}\label{sec:aim}

The aim of this Lab is to complete a set of exercises to demonstrate the understanding of the concepts of control systems. The exercises are based on the material covered in the lectures and the textbook. \\

Two design questions will also be completed, proving the understanding of the concepts of control systems, and the ability to apply them to real world problems.\\

Matlab will be used to complete the exercises and design questions. The ability to use Matlab will be demonstrated through the use of commands and functions required.
The code used to complete the exercises and design questions will be included in the report.\\

Problems will be approached both mathematically and graphically where appropriate. The graphs will be included in the report. \\

Analysis of system/plant/controller/feedback transfers functions for open and closed loop control systems will be demonstrated including the use of step response plots and root locus plots.\\

The concept of reducing transfer functions by analyzing the dominant roots is looked at and presented through the use of S-plane and Root locus plots.\\

%%%%%%%%%%%%%%%%%%%%%%%%%%%%%%%%%%%%%%%%%%%%%%%%%%%%%%%%%%%%%%%%%%%%%%%%%%%%%%%%%%%%%%%%%

\section{Approach}\label{sec:approach}

Firstly a project folder was created, to manage the workspace for the report and the matlab scripts. The necessary files were then added to the project folder including a title page and LaTex template. The files were then added to the git repository. The git repository was then pushed to github.\\

The report was written in LaTex using TeX Live 2019/Debian. The --shell-escape flag must be used when compiling the report to give the latex script access to shell commands. This allows the latex script to run matlab scripts from the terminal and input tikz pictures correctly .\\ 

The project required the use of matlab's control toolbox and matlab2tikz. The control toolbox and matlab2tikz were installed using the matlab package manager.\\

The matlab scripts were then written to produce the graphs required for the report. The matlab scripts were then integrated into the latex document using the matlab2tikz package.\\

Three primary matlab commands were used to create the graphs. step, rlocus and bode were used to create step response, root locus s-plane plots and bode diagrams respectively. The function matlab2tikz was then used to create a .tex file, to be included in the document.\\

The workspace as as above allows for any changes in the matlabs code, to be reflected in the report at run time, including automatically updating the graphs and appendix.\\

All figures, graphs, equations and tables were numbered, referenced and hyperlinked in the report, for easy viewing. A contents page was automatically generated, including sections, subsections, appendix and references.\\

For each exercise, the relevant mathematical notation for each system was written out in the report. The instructions were accurately followed to produce the necessary output. The graphs were included in the report using matlab2tikz. The Matlab code was then linked to the appendix. Further comments and explanations were then added including graphs, equations and tables where appropriate. \\

For each Question, the initial constants were calculated from the given information in the question. Where the second question required simplification, justification and explanation was given. The problems were first solved mathematically then checked using a bode plot to ensure the correct answer was obtained graphically. The final answer was then compared against the initial specification using a step response plot of the closed loop transfer function. Finally a bode plot was used to visualize the system and ascertain its bandwidth.  \\






\pagebreak
%%%%%%%%%%%%%%%%%%%%%%%%%%%%%%%%%%%%%%%%%%%%%%%%%%%%%%%%%%%%%%%%%%%%%%%%%%%%%%%%%%%%%%%%%
\section{Results and Discussion}\label{sec:results}

\subsection{Exercise 1}\label{sec:ex1}
\renewcommand{\ex}{ex1} 

\begin{equation}
    G_{p}\left(s\right)=\frac{a}{s+20}
    \label{eq:\ex}
\end{equation}

\begin{equation}
    \lim_{s\to0}\left(G_{p}\left(s\right)\right)
    \label{eq:\ex.1}
\end{equation}

\begin{figure}[ht!]
    \centering
    \input{matlab/Output/\ex.tex}
    \caption{Graph showing step response of plant. \appendixamble{\ex}}
    \label{fig:\ex}
\end{figure}\FloatBarrier

The plant transfer function for a simple first order system is given by \eqref{eq:\ex} where $a$ is the gain of the system and $s$ is the Laplace variable. Steady state is achieved when s approaches 0 as described by \eqref{eq:\ex.1}. Therefore the steady state gain of the system is $a/20$.

When $a=10$ the steady state gain is $\frac{10}{20}$ which is $0.5$.

Unity gain is achieved when the gain is equal to 1 at steady state. Hence the gain of the system is equal to 1 when $a=20$. The step response of such system is shown in Figure~\ref{fig:\ex}.

The time constant is the time taken for the system to reach $1-e^{-1}$ or approximately 63.2\% of its final value. The time constant is equal to the reciprocal of the denominator at steady state. Hence the time constant is equal to 0.05 at all values of $a$.





\pagebreak
\subsection{Exercise 2}\label{sec:ex2}
\renewcommand{\ex}{ex2} 

\begin{equation}
    G_{p}\left(s\right)=\frac{a}{s+a}
    \label{eq:\ex}
\end{equation}

\begin{figure}[ht!]
    \centering
    \input{matlab/Output/\ex.tex}
    \caption{Graph showing step response of the transfer function shown in \eqref{eq:\ex} where a = 1 (blue), 3 (red), 12 (green) and 20 (black). \appendixamble{\ex}} 
    \label{fig:\ex}
\end{figure}\FloatBarrier

\begin{table}[ht!]
    \centering
    \begin{tabular}{|c|c|}
        \hline
        a{[}arb{]} & Time Constant{[}s{]} \\ \hline
        1          & 1                    \\ \hline
        3          & 0.333                \\ \hline
        12         & 0.0883               \\ \hline
        20         & 0.05                 \\ \hline
    \end{tabular}
    \caption{Table to show the time constant of the system for different values of a}
    \label{tab:\ex}
\end{table}\FloatBarrier

Figure~\ref{fig:\ex}. shows the step response of the system for different values of $a$. The steady state gain of the system is $a/a$ which is equal to 1 for all values of $a$. The time constant of the system is shown in Table~\ref{tab:\ex}. The time constant is equal to the reciprocal of the denominator at steady state. Hence the time constant is equal to $1/a$ at all values of $a$.\\

\begin{figure}[ht!]
    \centering
    \input{matlab/Output/\ex_2.tex}
    \caption{Graph showing the step response of the transfer function shown in \eqref{eq:\ex_2}.\appendixamble{\ex}}
    \label{fig:\ex_2}
\end{figure}\FloatBarrier

\begin{equation}
    G_{p}\left(s\right)=\frac{a}{s-a}
    \label{eq:\ex_2}
\end{equation}

Figure~\ref{fig:\ex_2}. shows the step response of the transfer function shown in \eqref{eq:\ex_2}. where the plant has a positive pole the system is unstable and will not converge to a steady state. the rate of change of the system is increasing and will continue to increase until the system is destroyed. 


\subsection{Exercise 3}\label{sec:ex3}
\renewcommand{\ex}{ex3}

\begin{equation}
	G_{p}\left(s\right)=\frac{10a}{\left(s+10\right)\left(s+a\right)}
	\label{eq:\ex}
\end{equation}

\begin{figure}[ht!]
	\centering
	\input{matlab/Output/\ex.tex}
	\caption{Graph showing step response of the transfer function shown in \eqref{eq:\ex} where a = 1 (blue), 10 (red), 20 (green) and 100 (black).\appendixamble{\ex}}
	\label{fig:\ex}
\end{figure}\FloatBarrier


\begin{table}[ht!]
	\centering
	\begin{tabular}{|c|c|}
		\hline
		a{[}arb{]} & Time Constant $\tau${[}s{]} \\ \hline
		1          & 1.1                         \\ \hline
		10         & 0.215                       \\ \hline
		20         & 0.159                       \\ \hline
		100        & 0.111                       \\ \hline
	\end{tabular}
    \caption{Table to show the time constant of the system \eqref{eq:\ex} for different values of a}\label{table:\ex}
\end{table}\FloatBarrier

Figure~\ref{fig:\ex}. shows the step response of the system for different values of $a$. The steady state gain of the system is $10a/10a$ which is equal to 1 for all values of $a$. The time constant of the system is shown in Table~\ref{table:\ex}. When a is very small, the pole at -a is considered dominant, therefore the time constant approaches the reciprocal of -a. When a is large, the pole at -10 is considered dominant and therefore the time constant is approaches the reciprocal of -10. \\

\subsection{Exercise 4a}\label{sec:ex4a}
\renewcommand{\ex}{ex4a} 

\begin{figure}[ht!]
    \centering
    \input{matlab/Output/\ex.tex}
    \caption{\appendixamble{\ex}}
    \label{fig:\ex}
\end{figure}\FloatBarrier
\subsection{Exercise 4b}\label{sec:ex4b}
\renewcommand{\ex}{ex4b} 

\begin{figure}[ht!]
    \centering
    \input{matlab/Output/\ex.tex}
    \caption{\appendixamble{\ex}}
    \label{fig:\ex}
\end{figure}\FloatBarrier
\subsection{Exercise 5}\label{sec:ex5}
\renewcommand{\ex}{ex5} 

\begin{equation}
	G_{p}\left(s\right)=\frac{10}{s + 10}
	\label{eq:\ex}
\end{equation}

\begin{equation}
    \begin{split}
        G\left(s\right) &= G_{p}\left(s\right)\cdot G_{c}\left(s\right)\\
        &=\frac{k \cdot 10}{s + 10}
        \begin{cases}
            G_{c}\left(s\right)=k
        \end{cases}\\
    \end{split}
    \label{eq:\ex_1}
\end{equation}

\begin{equation}
    \begin{split}
        G\left(s\right) &= G_{p}\left(s\right)\cdot G_{c}\left(s\right)\\
        &=\frac{10}{s + 10} \cdot \frac{k\left(s+a\right)}{s}
        \begin{cases}
            G_{c}\left(s\right)=\frac{k\left(s+a\right)}{s}
        \end{cases}\\
        &=\frac{10k\left(s+205\right)}{s\left(s+10\right)}
        \begin{cases}
            a=205
        \end{cases}\\
    \end{split}
    \label{eq:\ex_2}
\end{equation}

\begin{figure}[ht!]
    \centering
    \input{matlab/Output/\ex.tex}
    \caption{Root Locus plot of the forward path transfer function \eqref{eq:\ex_1} of a proportional controller applied to plant \eqref{eq:\ex} \appendixamble{\ex}}
    \label{fig:\ex}
\end{figure}\FloatBarrier

Figure~\ref{fig:\ex} only has one pole on the real axis, which is at $s = -10$. This means that the system is stable. The root locus plot shows that the system is stable for all positive values of $k$. However it does not pass through out design point at $-200 \pm j200$ and does not meet the design specification. \\

Therefore in order to meet the design specification we need a second order controller for example a proportional integral controller.


\begin{figure}[ht!]
    \centering
    \input{matlab/Output/\ex_1.tex}
    \caption{Root Locus plot of the forward path transfer function \eqref{eq:\ex_2} of a proportional Integral controller applied to plant \eqref{eq:\ex} \appendixamble{\ex}}
    \label{fig:\ex_1}
\end{figure}\FloatBarrier

Figure~\ref{fig:\ex_1} shows that the system is stable for all positive values of $k$ and passes through the design point at $-200 \pm j200$, where k = 39.1. This means that the system meets the design specification and is the same value for k as mathematically calculated.

\begin{figure}[ht!]
    \centering
    \input{matlab/Output/\ex_2.tex}
    \caption{Step Response of the forward path transfer function \eqref{eq:\ex_2} of a proportional Integral controller applied to plant \eqref{eq:\ex} \appendixamble{\ex}}
    \label{fig:\ex_2}
\end{figure}\FloatBarrier

\begin{equation}
    \begin{split}
        \omega_{0} &= \frac{-\ln(\frac{\%}{100})}{t\cdot\xi}\\
        &=8.69338\ rad^s{-1}    \begin{cases}
            \%=2 \\
            t=0.75s \\
            \xi=0.6\\
        \end{cases}
    \end{split}\label{eq:\ex_2}
\end{equation}


Figure~\ref{fig:\ex_2} shows the step response of the system. The system is stable and the output is as expected. at 20ms the amplitude is 0.948. this is outside of the specified response of $\pm 2\%$ at 20ms. At 27.3ms the response is within the $\pm 2\%$. This is because the equation \eqref{eq:\ex_2} used to calulate $\omega_0$ is based off a first order decay. The system is second order therefore is only approximated by this equation.
\pagebreak
%%%%%%%%%%%%%%%%%%%%%%%%%%%%%%%%%%%%%%%%%%%%%%%%%%%%%%%%%%%%%%%%%%%%%%%%%%%%%%%%%%%%%%%%%

\section{Design Questions and Solutions}\label{sec:Q4}
\subsection{1}
\begin{equation}
\begin{split}
    0&=s^{2}+2\omega_{0}\xi s\ +\omega_{0}^{2}    \\
    \therefore s&=-13.5\pm j\frac{3\sqrt{19}}{2}
\end{split}\label{eq:Q4_1_1}
\end{equation}

\begin{conditions}
    \omega_0 & 15 \\
    \xi & 0.9 \\
\end{conditions}\\

Equation~\eqref{eq:Q4_1_1} shows the calculation for the approximate position of the dominant poles based on the values of $\omega_0$ and $\xi$.

\begin{equation}
    \begin{split}
        G_{c}\left(s\right)=\frac{k\left(s+a\right)}{s} \hspace{1cm} G_{p}\left(s\right)=\frac{6}{15s+1}
    \end{split}\label{eq:Q4_1_2}
\end{equation}

\begin{conditions}
    k & Constant of proportionality \\
    a & Constant of integration \\
\end{conditions}\\

\begin{equation}
    \begin{split}
        0&=1\ +G_{p}\left(s\right)\cdot G_{c}\left(s\right)\\
        &=s^{2}+2\omega_{0}\xi s\ +\omega_{0}^{2}\\
        &=s^{2}+\frac{\left(6k+1\right)}{15}s+\frac{6ka}{15} \\
    \end{split}\label{eq:Q4_1_3}
\end{equation}

The plant and general model for a PI controller are given by Eq.~\eqref{eq:Q4_1_2}. Setting the denominator of the system to zero, we re-arrange the controller to fit the general equation Eq.~\eqref{eq:Q4_1_3}.

\begin{equation}
    \begin{split}
        2\omega_{0}\xi &= \frac{\left(6k+1\right)}{15} \\
        \therefore k &= 67.33\\
    \end{split}\label{eq:Q4_1_4}
\end{equation}

\begin{conditions}
    \omega_0 & 15 \\
    \xi & 0.9 \\
\end{conditions}\\

\begin{equation}
    \begin{split}
        \omega_{0}^{2} &= \frac{6ka}{15}\\
        \therefore a &= 8.354 \\
    \end{split}\label{eq:Q4_1_5}
\end{equation}

\begin{conditions}
    \omega_0 & 15 \\
    k & 67.33 \\
\end{conditions}\\

Equation~\eqref{eq:Q4_1_4}-\eqref{eq:Q4_1_5} shows the calculation to find the constants of proportionality and integration $k$ and $a$ by comparing coefficients using Eq.~\eqref{eq:Q4_1_3} such that the system meets the specification set out by $\omega_0$ and $\xi$.

\begin{equation}
    \begin{split}
        G\left(s\right)= \frac{6}{15s+1}\cdot\frac{K\left(s+8.354\right)}{s}\\
    \end{split}\label{eq:Q4_1_6}
\end{equation}

\renewcommand{\ex}{Q4_1} 

\begin{figure}[ht!]
    \centering
    \input{matlab/Output/\ex.tex}
    \caption{Root Locus plot of the forward path transfer function \eqref{eq:Q4_1_6} \appendixamble{\ex}}
    \label{fig:\ex}
\end{figure}\FloatBarrier

Figure~\ref{fig:\ex} shows the root locus plot passing through the poles at $-13.5\pm j\frac{3\sqrt{19}}{2}$ as found in Eq.~\eqref{eq:Q4_1_1}. The gain at  $-13.5\pm j\frac{3\sqrt{19}}{2}$ is 67.33 confirming the value found in Eq.~\eqref{eq:Q4_1_4} to be correct.

\begin{equation}
    \begin{split}
        G_c\left(s\right)=\frac{67.33\left(s+8.354\right)}{s}\\
    \end{split}\label{eq:Q4_1_7}
\end{equation}

\begin{figure}[ht!]
    \centering
    \input{matlab/Output/\ex_1.tex}
    \caption{Step Response of the closed loop transfer function where the plant is given by Eq.~\eqref{eq:Q4_1_2} and controller Eq.~\eqref{eq:Q4_1_7}. \appendixamble{\ex}}
    \label{fig:\ex_1}
\end{figure}\FloatBarrier

Figure~\ref{fig:\ex_1} shows the step response of the closed loop transfer function. The step response is shown to be within the specification of $\omega_0 = 15$ and $\xi = 0.9$.\\

The final solution to this question for the design of a PI controller is given by Eq.~\eqref{eq:Q4_1_7}.

\begin{figure}[ht!]
    \centering
    \input{matlab/Output/\ex_2.tex}
    \input{matlab/Output/\ex_3.tex}
    \caption{Bode Diagram of the closed loop transfer function where the plant is given by Eq.~\eqref{eq:Q4_1_2} and controller Eq.~\eqref{eq:Q4_1_7}. \appendixamble{\ex}}
    \label{fig:\ex_2}
\end{figure}\FloatBarrier

\begin{figure}[ht!]
    \centering
    \inputminted[firstline=58,lastline=58]{matlab}{matlab/Q4_1.m}
    \caption{MATLAB code used to calculate the bandwidth of the closed loop system}\label{fig:Q4code1}
\end{figure}\FloatBarrier

Figure~\ref{fig:\ex_2} shows the Bode Diagram for the closed loop transfer function. The bandwidth of the closed loop system is defined as when the gain is above -3dB. The bandwidth is calculated on matlabs using figure~\ref{fig:Q4code1}. The result is 34.785Hz.\label{sec:Q4_1}
\subsection{2}
\begin{equation}
	\begin{split}
		\frac{\theta\left(s\right)}{V_{motor}\left(s\right)} &= \frac{15}{\left(s\ +\ 150\right)\left(s\ +\ 8\right)}\\
		&= \frac{15/150}{\left(s\ +\ 8\right)}\\
        &= \frac{0.1}{\left(s\ +\ 8\right)}
	\end{split}\label{eq:Q4_2_1}
\end{equation}

Equation~\eqref{eq:Q4_2_1} shows how the transfer function for the motor can be simplified from a third order equation to a second order equation. The root at -150 is several time larger than -8, therefore the root at -8 can be considered dominant and the transfer function can be simplified to a second order equation.

\begin{equation}
    \begin{split}
        \omega_{0} &= \frac{-\ln(\frac{\%}{100})}{t\cdot\xi}\\
        &=8.69338\ rad^s{-1}    \begin{cases}
            \%=2 \\
            t=0.75s \\
            \xi=0.6\\
        \end{cases}
    \end{split}\label{eq:Q4_2_2}
\end{equation}

Equation~\eqref{eq:Q4_2_2} shows the calculation to find the natural frequency of the system. The natural frequency is found by using the formula for the natural frequency of a second order system with damping ratio $\xi$ and time constant $t$.

\begin{equation}
    \begin{split}
        0&=s^{2}+2\omega_{0}\xi s\ +\omega_{0}^{2}\\
        \therefore s&=-5.216028\pm j6.954704
        \begin{cases}
            \omega_0 = 8.69338 \\
            \xi = 0.6 \\
        \end{cases}
    \end{split}\label{eq:Q4_2_3}
    \end{equation}

Equation~\eqref{eq:Q4_2_3} shows the calculation for the approximate position of the dominant poles based on the values of $\omega_0$ and $\xi$.

    \begin{equation}
        \begin{split}
            G_{c}\left(s\right)=\frac{k\left(s+a\right)}{s} \hspace{1cm} G_{p}\left(s\right)=\frac{0.1}{s+8}
        \end{split}\label{eq:Q4_2_4}
    \end{equation}
    
    \begin{equation}
        \begin{split}
            0&=1\ +G_{p}\left(s\right)\cdot G_{c}\left(s\right)\\
            &=s^{2}+2\omega_{0}\xi s\ +\omega_{0}^{2}\\
            &=s^{2\ }+\left(8+0.1k\right)s+0.1ka\\
        \end{split}\label{eq:Q4_2_5}
    \end{equation}

    The plant and general model for a PI controller are given by Eq.~\eqref{eq:Q4_2_4}. Setting the denominator of the system to zero, we re-arrange the controller to fit the general equation Eq.~\eqref{eq:Q4_2_5}.

    \begin{equation}
        \begin{split}
            2\omega_{0}\xi &= 8+0.1k\\
            \therefore k &= 24.32
            \begin{cases}
                \omega_0 = 8.6934 \\
                \xi = 0.6 \\
            \end{cases}\\
        \end{split}\label{eq:Q4_2_6}
    \end{equation}
    
    \begin{equation}
        \begin{split}
            \omega_{0}^{2} &= 0.1ka\\
            \therefore a &= 31.0752    
            \begin{cases}
                \omega_0 = 8.6934 \\
                k = 24.32 \\
            \end{cases}
        \end{split}\label{eq:Q4_2_7}
    \end{equation}

    Equation~\eqref{eq:Q4_2_6}-\eqref{eq:Q4_2_7} shows the calculation to find the constants of proportionality and integration $k$ and $a$ by comparing coefficients using Eq.~\eqref{eq:Q4_2_5} such that the system meets the specification set out by $\omega_0$ and $\xi$.

    \begin{equation}
        \begin{split}
            G_c\left(s\right)=\frac{24.32\left(s+31.0752\right)}{s}\\
        \end{split}\label{eq:Q4_2_8}
    \end{equation}

    Equation~\eqref{eq:Q4_2_8} shows the transfer function for the PI controller.

    \renewcommand{\ex}{Q4_2} 

\begin{figure}[ht!]
    \centering
    \input{matlab/Output/\ex.tex}
    \caption{\appendixamble{\ex}}
    \label{fig:\ex}
\end{figure}\FloatBarrier

\begin{figure}[ht!]
    \centering
    \input{matlab/Output/\ex_1.tex}
    \caption{\appendixamble{\ex}}
    \label{fig:\ex}
\end{figure}\FloatBarrier

\begin{figure}[ht!]
    \centering
    \input{matlab/Output/\ex_2.tex}
    \input{matlab/Output/\ex_3.tex}
    \caption{\appendixamble{\ex}}
    \label{fig:\ex}
\end{figure}\FloatBarrier

\begin{figure}[ht!]
    \centering
    \inputminted[firstline=58,lastline=58]{matlab}{matlab/Q4_1.m}
    \caption{MATLAB code used to calculate the bandwidth of the closed loop system}\label{fig:Q4code}
\end{figure}\FloatBarrier

The bandwidth of the closed loop system is there the gain is above -3dB. The bandwidth is calculated on matlabs using figure~\ref{fig:Q4code}. The result is 10.7049Hz.

\begin{equation}
    G_{p}= \frac{15}{s^{2}+158s+1200}
\end{equation}

\begin{equation}
    G_{c}=\frac{24.32s+755.7}{s}
\end{equation}

\begin{equation}
    G=\frac{364.8+11340}{s^{3}+158s^{2}+1200s}
\end{equation}

\begin{equation}
    H=\frac{364.8s+11340}{s^{3}+158s^{2}+1565+11340}
\end{equation}
\label{sec:Q4_2}
\pagebreak
%%%%%%%%%%%%%%%%%%%%%%%%%%%%%%%%%%%%%%%%%%%%%%%%%%%%%%%%%%%%%%%%%%%%%%%%%%%%%%%%%%%%%%%%%

\section{Summary and Conclusions}\label{sec:summary}

The aim of this lab was to complete a set of exercises to demonstrate the understanding of the concepts of control systems. The exercises were based on the material covered in the lectures and the textbook. Five exercises and two questions completed, proving the understanding of the concepts of control systems, and the ability to apply them to a real world problem.\\

Matlab was used to complete the exercises and design questions. The ability to use Matlab was demonstrated by the through the use of commands and functions required to produce the graphs for this document.

To conclude the lab was completed successfully, and the graphs produced were accurate and correct.\\


\pagebreak
%%%%%%%%%%%%%%%%%%%%%%%%%%%%%%%%%%%%%%%%%%%%%%%%%%%%%%%%%%%%%%%%%%%%%%%%%%%%%%%%%%%%%%%%%

%\addcontentsline{toc}{section}{References}
%\bibliographystyle{IEEEtranN}
%\bibliography{biblist}

%\pagebreak
%%%%%%%%%%%%%%%%%%%%%%%%%%%%%%%%%%%%%%%%%%%%%%%%%%%%%%%%%%%%%%%%%%%%%%%%%%%%%%%%%%%%%%%%%

\appendix
\section*{Appendices}\addcontentsline{toc}{section}{Appendices}\label{appendix:main}
\renewcommand{\thesubsection}{\Alph{subsection}}

\subsection{Matlab Code}
\subsubsection{Ex1: (\ref{sec:ex1})}\label{appendix:ex1}
\inputminted{matlab}{matlab/ex1.m}

\subsubsection{Ex2: (\ref{sec:ex2})}\label{appendix:ex2}
\inputminted{matlab}{matlab/ex2.m}

\subsubsection{Ex3: (\ref{sec:ex3})}\label{appendix:ex3}
\inputminted{matlab}{matlab/ex3.m}

\subsubsection{Ex4a: (\ref{sec:ex4a})}\label{appendix:ex4a}
\inputminted{matlab}{matlab/ex4a.m}

\subsubsection{Ex4b: (\ref{sec:ex4b})}\label{appendix:ex4b}
\inputminted{matlab}{matlab/ex4b.m}

\subsubsection{Ex5: (\ref{sec:ex5})}\label{appendix:ex5}
\inputminted{matlab}{matlab/ex5.m}

\subsubsection{Q4.1: (\ref{sec:Q4_1})}\label{appendix:Q4_1}
\inputminted{matlab}{matlab/Q4_1.m}

\subsubsection{Q4.2: (\ref{sec:Q4_2})}\label{appendix:Q4_2}
\inputminted{matlab}{matlab/Q4_2.m}
%%%%%%%%%%%%%%%%%%%%%%%%%%%%%%%%%%%%%%%%%%%%%%%%%%%%%%%%%%%%%%%%%%%%%%%%%%%%%%%%%%%%%%%%%

\end{document}