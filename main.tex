\documentclass[12pt]{article}

\usepackage[colorlinks=true, allcolors=blue]{hyperref}
\usepackage[english]{babel}
\usepackage[numbers]{natbib}
\usepackage{url}
\usepackage[utf8x]{inputenc}
\usepackage{amsmath}
\usepackage{graphicx}
\usepackage{parskip}
\usepackage{fancyhdr}
\usepackage{vmargin}
\usepackage{tikz}
\usepackage{pgfplots} 
\usepackage{minted}
\usepackage{placeins}


\graphicspath{{images/}}
\setmarginsrb{3 cm}{2.5 cm}{3 cm}{2.5 cm}{1 cm}{1.5 cm}{1 cm}{1.5 cm}
\usetikzlibrary{angles,quotes} % for pic (angle labels)
\usetikzlibrary{arrows.meta}
\usetikzlibrary{calc}

\title{EEEE2045 Control Coursework}								% Title
\author{George Downing}								                        % Author
\date{October 24, 2022}											            % Date

\makeatletter
\let\thetitle\@title
\let\theauthor\@author
\let\thedate\@date
\makeatother

\pagestyle{fancy}
\fancyhf{}
\rhead{\theauthor}
\lhead{\thetitle}
\cfoot{\thepage}



\begin{document}

%%%%%%%%%%%%%%%%%%%%%%%%%%%%%%%%%%%%%%%%%%%%%%%%%%%%%%%%%%%%%%%%%%%%%%%%%%%%%%%%%%%%%%%%%

\begin{titlepage}
    \centering
    \vspace*{0.5 cm}
    \includegraphics[scale = 0.4]{uon.png}\\[1.0 cm]	% University Logo
    \textsc{\LARGE University of Nottingham}\\[2.0 cm]	% University Name
    \textsc{\Large EEEE2045:}\\[0.5 cm]				% Course Code
    \textsc{\large Control Coursework}\\[0.5 cm]				% Course Name
    \rule{\linewidth}{0.2 mm} \\[0.4 cm]
    { \huge \bfseries \thetitle}\\
    \rule{\linewidth}{0.2 mm} \\[1.5 cm]

    \begin{minipage}{0.4\textwidth}
        \begin{flushleft} \large
            \emph{Author:}\\
            \theauthor
        \end{flushleft}
    \end{minipage}~
    \begin{minipage}{0.4\textwidth}
        \begin{flushright} \large
            \emph{Student Number:} \\
            20273662									% Your Student Number
        \end{flushright}
    \end{minipage}\\[2 cm]

    {\large \thedate}\\[2 cm]

    \vfill

\end{titlepage}

%%%%%%%%%%%%%%%%%%%%%%%%%%%%%%%%%%%%%%%%%%%%%%%%%%%%%%%%%%%%%%%%%%%%%%%%%%%%%%%%%%%%%%%%%

\tableofcontents
\pagebreak

%%%%%%%%%%%%%%%%%%%%%%%%%%%%%%%%%%%%%%%%%%%%%%%%%%%%%%%%%%%%%%%%%%%%%%%%%%%%%%%%%%%%%%%%%

\section{Abstract}
In the optical absorption of semiconductor experiment, we discovered the Urbach slope, the band gap energy for both; direct and indirect band gap semiconductor. The experiment was carried out using Matlab to automate the wavelengths by outputting a signal through a stepper motor to the monochromator and analysing the data inputted back in through plotting graphs. The Urbach slope was found to be $(0.182\pm0.008)$ eV, the band gap energy for a direct band gap semiconductor to be $(1.431\pm0.002)$ eV and the band gap energy for an indirect band gap semiconductor to be $(1.8305\pm0.0146)$ eV.


\section{Introduction}
testZ~\cite{pratt_2022}

\section{Theory}
hello123

\begin{table}[ht!]
    \centering
    \begin{tabular}{|l|l|}
        \hline
        Sign type            & Sign Observed \\ \hline
        Distance Measurement & yes           \\ \hline
        Green Short Cut      & yes           \\ \hline
        Red Short Cut        & yes           \\ \hline
        Blue Short Cut       & yes           \\ \hline
        Follow Black         & yes           \\ \hline
        IMU Measurement      & yes           \\ \hline
        Shape Counter        & yes           \\ \hline
        Stop Light           & yes           \\ \hline
        Yellow Short Cut     & yes           \\ \hline
    \end{tabular}
    \caption{Table to show success of identifying a sign within a frame}
    \label{tab:identifying a sign within a frame}
\end{table}

table \ref{tab:identifying a sign within a frame}


\subsection{Code Overview}
\begin{figure}[ht!]
\inputminted[firstline=12, lastline=50,fontsize=\scriptsize, tabsize=4]{c}{helloworld.c}
   
    \caption{Handler routine for the Arduino nano written to communicate to I2c with the EPS32 }
    \label{fig:i2chandler}
\end{figure}\FloatBarrier

\begin{figure}
    \centering
    % This file was created by matlab2tikz.
%
%The latest updates can be retrieved from
%  http://www.mathworks.com/matlabcentral/fileexchange/22022-matlab2tikz-matlab2tikz
%where you can also make suggestions and rate matlab2tikz.
%
\definecolor{mycolor1}{rgb}{0.00000,0.44700,0.74100}%
%
\begin{tikzpicture}

\begin{axis}[%
width=4.568in,
height=3.603in,
at={(0.766in,0.486in)},
scale only axis,
xmin=0,
xmax=2,
ymin=-0.8,
ymax=1,
axis background/.style={fill=white}
]
\addplot [color=mycolor1, forget plot]
  table[row sep=crcr]{%
0	0\\
0.05	0.00249999739583415\\
0.1	0.00999983333416667\\
0.15	0.0224981016105536\\
0.2	0.0399893341866342\\
0.25	0.0624593178423802\\
0.3	0.0898785491980111\\
0.35	0.122193852192663\\
0.4	0.159318206614246\\
0.45	0.201118873846073\\
0.5	0.247403959254523\\
0.55	0.297907621896134\\
0.6	0.35227423327509\\
0.65	0.410041898781764\\
0.7	0.470625888171158\\
0.75	0.53330267353602\\
0.8	0.597195441362392\\
0.85	0.661262123760472\\
0.9	0.724287174370143\\
0.95	0.784878485034067\\
1	0.841470984807897\\
1.05	0.89233856416221\\
1.1	0.935616001553386\\
1.15	0.969332510867995\\
1.2	0.991458348191686\\
1.25	0.999965585678249\\
1.3	0.992903651094119\\
1.35	0.968489520283355\\
1.4	0.925211520788168\\
1.45	0.861944555174211\\
1.5	0.778073196887921\\
1.55	0.673617587361253\\
1.6	0.549355436427126\\
1.65	0.406931797351064\\
1.7	0.248946786673153\\
1.75	0.0790102167473897\\
1.8	-0.0982485937451087\\
1.85	-0.277227544887737\\
1.9	-0.451465752161423\\
1.95	-0.613833396107781\\
2	-0.756802495307928\\
};
\end{axis}
\end{tikzpicture}%
    \caption{test}
    \label{fig:test}
\end{figure}\FloatBarrier


\bibliographystyle{IEEEtranN}
\bibliography{biblist}

\end{document}