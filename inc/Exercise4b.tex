\renewcommand{\ex}{ex4b} 

\begin{equation}
	G_{p}\left(s\right)=\frac{100a}{\left(s^2 + 12s + 100\right)\left(s+a\right)}
	\label{eq:\ex}
\end{equation}

\begin{figure}[ht!]
    \centering
    \input{matlab/Output/\ex.tex}
    \caption{Graph showing step response of the transfer function shown in \eqref{eq:\ex} where a = 5 (red), 7 (green), 10 (blue), 25 (black) and 35 (magenta). \appendixamble{\ex}}
    \label{fig:\ex}
\end{figure}\FloatBarrier

\begin{figure}[ht!]
    \centering
    \input{matlab/Output/\ex_1.tex}
    \caption{Graph showing Pole-Zero Map of the transfer function shown in \eqref{eq:\ex} where a = 5 (red), 7 (green), 10 (blue), 25 (black) and 35 (magenta), The poles at -6$\pm j8$ exist for all values of a. \appendixamble{\ex}}
    \label{fig:\ex_2}
\end{figure}\FloatBarrier

Figure~\ref{fig:\ex}. shows the step response of the system \eqref{eq:\ex} for different values of $a$. The steady state gain of the system is $100a/100a$ which is equal to 1 for all values of $a$. The poles of the system are shown in Figure~\ref{fig:\ex_2}. The poles at -6$\pm j8$ exist for all values of $a$. The poles at -a and -6$\pm j8$ are considered dominant for small and large values of $a$ respectively. \\

When a is 35, the poles at -6$\pm j8$ are considered almost completely dominant and therefore the graph looks like an underdamped third order system. When a is 5, the pole at -a is slightly more dominant over the poles at -6$\pm j8$ therefore the graph looks like a critically damped second order system. The extent to which the systems order can be approximated, can be visualized with the help of the pole-zero map shown in Figure~\ref{fig:\ex_2}. The ratio between real parts of each pole determines how dominant it is, where the smaller the ratio the more dominant the pole is. 
