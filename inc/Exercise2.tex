\renewcommand{\ex}{ex2} 

\begin{equation}
    G_{p}\left(s\right)=\frac{a}{s+a}
    \label{eq:\ex}
\end{equation}

\begin{figure}[ht!]
    \centering
    \input{matlab/Output/\ex.tex}
    \caption{Graph showing step response of the transfer function shown in \eqref{eq:\ex} where a = 1 (blue), 3 (red), 12 (green) and 20 (black). \appendixamble{\ex}} 
    \label{fig:\ex}
\end{figure}\FloatBarrier

\begin{table}[ht!]
    \centering
    \begin{tabular}{|c|c|}
        \hline
        a{[}arb{]} & Time Constant{[}s{]} \\ \hline
        1          & 1                    \\ \hline
        3          & 0.333                \\ \hline
        12         & 0.0883               \\ \hline
        20         & 0.05                 \\ \hline
    \end{tabular}
    \caption{Table to show the time constant of the system for different values of a}
    \label{tab:\ex}
\end{table}\FloatBarrier

Figure~\ref{fig:\ex}. shows the step response of the system for different values of $a$. The steady state gain of the system is $a/a$ which is equal to 1 for all values of $a$. The time constant of the system is shown in Table~\ref{tab:\ex}. The time constant is equal to the reciprocal of the denominator at steady state. Hence the time constant is equal to $1/a$ at all values of $a$.\\

\begin{figure}[ht!]
    \centering
    \input{matlab/Output/\ex_2.tex}
    \caption{Graph showing the step response of the transfer function shown in \eqref{eq:\ex_2}. \appendixamble{\ex}}
    \label{fig:\ex_2}
\end{figure}\FloatBarrier

\begin{equation}
    G_{p}\left(s\right)=\frac{a}{s-a}
    \label{eq:\ex_2}
\end{equation}

Figure~\ref{fig:\ex_2}. shows the step response of the transfer function shown in \eqref{eq:\ex_2}. Where the plant has a positive pole the system is unstable and will not converge to a steady state. The rate of change of the system is increasing and will continue to increase until the system is destroyed. 

