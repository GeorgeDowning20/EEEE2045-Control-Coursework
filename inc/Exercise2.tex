
\begin{equation}
    G_{p}\left(s\right)=\frac{a}{s+a}
    \label{eq:ex2}
\end{equation}

\begin{figure}[ht!]
    \centering
    \input{matlab/Ex2.tex}
    \caption{Graph showing step response of the transfer function shown in \eqref{eq:ex2} where a = 1 (blue), 3 (red), 12 (green) and 20 (black) }
    \label{fig:ex2}
\end{figure}\FloatBarrier

\begin{table}[ht!]
    \centering
    \begin{tabular}{|c|c|}
        \hline
        a{[}arb{]} & Time Constant{[}s{]} \\ \hline
        1          & 1                    \\ \hline
        3          & 0.333                \\ \hline
        12         & 0.0883               \\ \hline
        20         & 0.05                 \\ \hline
    \end{tabular}
    \caption{Table to show the time constant of the system for different values of a}
    \label{tab:ex2}
\end{table}\FloatBarrier

Figure~\ref{fig:ex2}. shows the step response of the system for different values of $a$. The steady state gain of the system is $a/a$ which is equal to 1 for all values of $a$. The time constant of the system is shown in Table~\ref{tab:ex2}. The time constant is equal to the reciprocal of the denominator at steady state. Hence the time constant is equal to $1/a$ at all values of $a$.\\

\begin{figure}[ht!]
    \centering
    % This file was created by matlab2tikz.
%
%The latest updates can be retrieved from
%  http://www.mathworks.com/matlabcentral/fileexchange/22022-matlab2tikz-matlab2tikz
%where you can also make suggestions and rate matlab2tikz.
%
\definecolor{mycolor1}{rgb}{0.00000,0.44700,0.74100}%
%
\begin{tikzpicture}

\begin{axis}[%
width=4.442in,
height=3.393in,
at={(0.893in,0.486in)},
scale only axis,
separate axis lines,
every outer x axis line/.append style={white!40!black},
every x tick label/.append style={font=\color{white!40!black}},
every x tick/.append style={white!40!black},
xmin=0,
xmax=1,
every outer y axis line/.append style={white!40!black},
every y tick label/.append style={font=\color{white!40!black}},
every y tick/.append style={white!40!black},
ymin=0,
ymax=500000,
axis background/.style={fill=white}
]
\addplot [color=mycolor1, forget plot]
  table[row sep=crcr]{%
0	0\\
0.00460517016472295	0.0964781961520202\\
0.00921034038765356	0.202264434599783\\
0.0138155105523765	0.318256738537457\\
0.0184206807170995	0.445439770759549\\
0.0230258509400301	0.584893192455638\\
0.027631021104753	0.737800828763284\\
0.0322361913276836	0.905460717971437\\
0.0368413614924066	1.08929613087093\\
0.0414465316571295	1.29086765274405\\
0.0460517018800601	1.51188643148635\\
0.0506568720447831	1.75422870332841\\
0.055262042209506	2.01995172043098\\
0.0598672124324366	2.31131121481303\\
0.0644723825971596	2.63078054768266\\
0.0690775527618825	2.98107170552248\\
0.0736827229848132	3.36515832238365\\
0.0782878931495361	3.78630092320964\\
0.0828930633724667	4.24807460251031\\
0.0874982335371897	4.75439937337069\\
0.0921034037019126	5.30957344482886\\
0.0967085739248432	5.91830970917363\\
0.101313744089566	6.58577575028175\\
0.105918914254289	7.31763771100668\\
0.11052408447722	8.1201083935448\\
0.115129254641943	9\\
0.119734424806666	9.96478196140379\\
0.124339595029596	11.0226443461725\\
0.128944765194319	12.1825673855492\\
0.13354993541725	13.4543977074791\\
0.138155105581973	14.8489319246146\\
0.142760275746696	16.3780082875164\\
0.147365445969626	18.0546071796562\\
0.151970616134349	19.8929613085347\\
0.156575786299072	21.9086765276734\\
0.161180956522003	24.1188643150963\\
0.165786126686726	26.5422870334005\\
0.170391296909656	29.1995172040188\\
0.174996467074379	32.1131121482467\\
0.179601637239102	35.3078054770012\\
0.184206807462033	38.8107170553412\\
0.188811977626756	42.6515832240111\\
0.193417147791479	46.863009232271\\
0.198022318014409	51.4807460249867\\
0.202627488179132	56.5439937337069\\
0.207232658343855	62.0957344479975\\
0.211837828566786	68.1830970919109\\
0.216442998731509	74.857757502934\\
0.221048168954439	82.1763771102414\\
0.225653339119162	90.2010839355644\\
0.230258509283885	99\\
0.234863679506816	108.647819614329\\
0.239468849671539	119.226443461725\\
0.244074019836262	130.825673855608\\
0.248679190059192	143.543977074558\\
0.253284360223915	157.489319246088\\
0.257889530388638	172.780082874931\\
0.262494700611569	189.546071796329\\
0.267099870776292	207.929613085405\\
0.271705040999223	228.086765276734\\
0.276310211163945	250.188643150963\\
0.280915381328668	274.422870333772\\
0.285520551551599	300.995172040188\\
0.290125721716322	330.131121482584\\
0.294730891881045	362.078054770071\\
0.299336062103976	397.10717055347\\
0.303941232268699	435.515832240111\\
0.308546402433421	477.630092322594\\
0.313151572656352	523.807460249751\\
0.317756742821075	574.439937337127\\
0.322361913044006	629.95734448015\\
0.326967083208729	690.830970918876\\
0.331572253373452	757.577575029107\\
0.336177423596382	830.763771102589\\
0.340782593761105	911.010839355818\\
0.345387763925828	998.999999999942\\
0.349992934148759	1095.47819614311\\
0.354598104313482	1201.2644346173\\
0.359203274478205	1317.25673855632\\
0.363808444701135	1444.43977074581\\
0.368413614865858	1583.89319246099\\
0.373018785088789	1736.8008287492\\
0.377623955253512	1904.46071796305\\
0.382229125418235	2088.29613085382\\
0.386834295641165	2289.86765276757\\
0.391439465805888	2510.88643150934\\
0.396044635970611	2753.2287033379\\
0.400649806193542	3018.9517204017\\
0.405254976358265	3310.3112148256\\
0.409860146581195	3629.78054770065\\
0.414465316745918	3980.07170553459\\
0.419070486910641	4364.15832240123\\
0.423675657133572	4785.30092322588\\
0.428280827298295	5247.07460249716\\
0.432885997463018	5753.39937337098\\
0.437491167685948	6308.57344480127\\
0.442096337850671	6917.30970918864\\
0.446701508015394	7584.77575029101\\
0.451306678238325	8316.63771102577\\
0.455911848403048	9119.10839355807\\
0.460517018625978	9998.99999999889\\
0.465122188790701	10963.7819614306\\
0.469727358955424	12021.6443461727\\
0.474332529178355	13181.5673855626\\
0.478937699343078	14453.3977074576\\
0.483542869507801	15847.9319246093\\
0.488148039730731	17377.0082874917\\
0.492753209895454	19053.6071796302\\
0.497358380060177	20891.9613085379\\
0.501963550283108	22907.676527675\\
0.506568720447831	25117.8643150927\\
0.511173890670761	27541.2870333783\\
0.515779060835484	30198.5172040164\\
0.520384231000207	33112.1121482549\\
0.524989401223138	36306.8054770055\\
0.529594571387861	39809.7170553446\\
0.534199741552584	43650.5832240109\\
0.538804911775514	47862.0092322576\\
0.543410081940237	52479.7460249703\\
0.54801525210496	57542.993733708\\
0.552620422327891	63094.7344480109\\
0.557225592492614	69182.0970918843\\
0.561830762715545	75856.757502908\\
0.566435932880268	83175.3771102556\\
0.57104110304499	91200.0839355783\\
0.575646273267921	99998.999999986\\
0.580251443432644	109646.819614303\\
0.584856613597367	120225.443461724\\
0.589461783820298	131824.673855622\\
0.594066953985021	144542.977074572\\
0.598672124149743	158488.319246088\\
0.603277294372674	173779.082874912\\
0.607882464537397	190545.071796297\\
0.612487634760328	208928.613085373\\
0.617092804925051	229085.765276743\\
0.621697975089774	251187.64315092\\
0.626303145312704	275421.870333775\\
0.630908315477427	301994.172040155\\
0.63551348564215	331130.12148254\\
0.640118655865081	363077.054770045\\
0.644723826029804	398106.170553435\\
0.649328996252734	436514.832240097\\
0.653934166417457	478629.092322562\\
0.656236751528922	501186.233627192\\
};
\end{axis}

\begin{axis}[%
width=4.568in,
height=3.603in,
at={(0.766in,0.486in)},
scale only axis,
xmin=0,
xmax=1,
xtick={\empty},
xlabel={Time (seconds)},
ymin=0,
ymax=1,
ytick={\empty},
ylabel={Amplitude},
axis line style={draw=none},
ticks=none,
title style={font=\bfseries},
title={Step Response},
axis x line*=bottom,
axis y line*=left
]
\end{axis}
\end{tikzpicture}%
    \caption{graph showing the step response of the transfer function shown in \eqref{eq:ex2_2} }
    \label{fig:ex2_2}
\end{figure}\FloatBarrier

\begin{equation}
    G_{p}\left(s\right)=\frac{a}{s-a}
    \label{eq:ex2_2}
\end{equation}

Figure~\ref{fig:ex2_2}. shows the step response of the transfer function shown in \eqref{eq:ex2_2}. where the plant has a positive pole the system is unstable and will not converge to a steady state. the rate of change of the system is increasing and will continue to increase until the system is destroyed. 

