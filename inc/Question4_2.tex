\begin{equation}
	\begin{split}
		\frac{\theta\left(s\right)}{V_{motor}\left(s\right)} &= \frac{15}{\left(s\ +\ 150\right)\left(s\ +\ 8\right)}\\
		&= \frac{15/150}{\left(s\ +\ 8\right)}\\
        &= \frac{0.1}{\left(s\ +\ 8\right)}
	\end{split}\label{eq:Q4_2_1}
\end{equation}

Equation~\eqref{eq:Q4_2_1} shows how the transfer function for the motor can be simplified from a third order equation to a second order equation. The root at -150 is several time larger than -8, therefore the root at -8 can be considered dominant and the transfer function can be simplified to a second order equation.

\begin{equation}
    \begin{split}
        \omega_{0} &= \frac{-\ln(\frac{\%}{100})}{t\cdot\xi}\\
        &=8.69338\ rad^s{-1}    \begin{cases}
            \%=2 \\
            t=0.75s \\
            \xi=0.6\\
        \end{cases}
    \end{split}\label{eq:Q4_2_2}
\end{equation}

Equation~\eqref{eq:Q4_2_2} shows the calculation to find the natural frequency of the system. The natural frequency is found by using the formula for the natural frequency of a second order system with damping ratio $\xi$ and time constant $t$.

\begin{equation}
    \begin{split}
        0&=s^{2}+2\omega_{0}\xi s\ +\omega_{0}^{2}\\
        \therefore s&=-5.216028\pm j6.954704
        \begin{cases}
            \omega_0 = 8.69338 \\
            \xi = 0.6 \\
        \end{cases}
    \end{split}\label{eq:Q4_2_3}
    \end{equation}

Equation~\eqref{eq:Q4_2_3} shows the calculation for the approximate position of the dominant poles based on the values of $\omega_0$ and $\xi$.

    \begin{equation}
        \begin{split}
            G_{c}\left(s\right)=\frac{k\left(s+a\right)}{s} \hspace{1cm} G_{p}\left(s\right)=\frac{0.1}{s+8}
        \end{split}\label{eq:Q4_2_4}
    \end{equation}
    
    \begin{equation}
        \begin{split}
            0&=1\ +G_{p}\left(s\right)\cdot G_{c}\left(s\right)\\
            &=s^{2}+2\omega_{0}\xi s\ +\omega_{0}^{2}\\
            &=s^{2\ }+\left(8+0.1k\right)s+0.1ka\\
        \end{split}\label{eq:Q4_2_5}
    \end{equation}

    The plant and general model for a PI controller are given by Eq.~\eqref{eq:Q4_2_4}. Setting the denominator of the system to zero, we re-arrange the controller to fit the general equation Eq.~\eqref{eq:Q4_2_5}.

    \begin{equation}
        \begin{split}
            2\omega_{0}\xi &= 8+0.1k\\
            \therefore k &= 24.32
            \begin{cases}
                \omega_0 = 8.6934 \\
                \xi = 0.6 \\
            \end{cases}\\
        \end{split}\label{eq:Q4_2_6}
    \end{equation}
    
    \begin{equation}
        \begin{split}
            \omega_{0}^{2} &= 0.1ka\\
            \therefore a &= 31.0752    
            \begin{cases}
                \omega_0 = 8.6934 \\
                k = 24.32 \\
            \end{cases}
        \end{split}\label{eq:Q4_2_7}
    \end{equation}

    Equation~\eqref{eq:Q4_2_6}-\eqref{eq:Q4_2_7} shows the calculation to find the constants of proportionality and integration $k$ and $a$ by comparing coefficients using Eq.~\eqref{eq:Q4_2_5} such that the system meets the specification set out by $\omega_0$ and $\xi$.

    \begin{equation}
        \begin{split}
            G_c\left(s\right)=\frac{24.32\left(s+31.0752\right)}{s}\\
        \end{split}\label{eq:Q4_2_8}
    \end{equation}

    Equation~\eqref{eq:Q4_2_8} shows the transfer function for the PI controller.

    \renewcommand{\ex}{Q4_2} 

\begin{figure}[ht!]
    \centering
    \input{matlab/Output/\ex.tex}
    \caption{\appendixamble{\ex}}
    \label{fig:\ex}
\end{figure}\FloatBarrier

\begin{figure}[ht!]
    \centering
    \input{matlab/Output/\ex_1.tex}
    \caption{\appendixamble{\ex}}
    \label{fig:\ex}
\end{figure}\FloatBarrier

\begin{figure}[ht!]
    \centering
    \input{matlab/Output/\ex_2.tex}
    \input{matlab/Output/\ex_3.tex}
    \caption{\appendixamble{\ex}}
    \label{fig:\ex}
\end{figure}\FloatBarrier

\begin{figure}[ht!]
    \centering
    \inputminted[firstline=58,lastline=58]{matlab}{matlab/Q4_1.m}
    \caption{MATLAB code used to calculate the bandwidth of the closed loop system}\label{fig:Q4code}
\end{figure}\FloatBarrier

The bandwidth of the closed loop system is there the gain is above -3dB. The bandwidth is calculated on matlabs using figure~\ref{fig:Q4code}. The result is 10.7049Hz.

\begin{equation}
    G_{p}= \frac{15}{s^{2}+158s+1200}
\end{equation}

\begin{equation}
    G_{c}=\frac{24.32s+755.7}{s}
\end{equation}

\begin{equation}
    G=\frac{364.8+11340}{s^{3}+158s^{2}+1200s}
\end{equation}

\begin{equation}
    H=\frac{364.8s+11340}{s^{3}+158s^{2}+1565+11340}
\end{equation}
