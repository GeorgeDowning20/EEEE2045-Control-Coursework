\begin{equation}
	\begin{split}
		\frac{\theta\left(s\right)}{V_{motor}\left(s\right)} &= \frac{15}{\left(s\ +\ 150\right)\left(s\ +\ 8\right)}\\
		&= \frac{15/150}{\left(s\ +\ 8\right)}\\
		&= \frac{0.1}{\left(s\ +\ 8\right)}
	\end{split}\label{eq:Q4_2_1}
\end{equation}

Equation~\eqref{eq:Q4_2_1} shows how the transfer function for the motor can be simplified from a third order equation to a second order equation. The root at -150 is several times larger than -8, therefore the root at -8 can be considered dominant and the transfer function can be simplified to a second order equation.

\begin{equation}
	\begin{split}
		\omega_{0} &= \frac{-\ln(\frac{\%}{100})}{t\cdot\xi}\\
		&=8.69338\ rads^{-1}   \\
	\end{split}\label{eq:Q4_2_2}
\end{equation}

\begin{conditions}
	\% & 2 \\
	t & 0.75s \\
	\xi & 0.6\\
\end{conditions}\\

Equation~\eqref{eq:Q4_2_2} shows the calculation to find the natural frequency of the system. The natural frequency is found by using the formula for the natural frequency of a second order system with damping ratio $\xi$ and time constant $t$.

\begin{equation}
	\begin{split}
		0&=s^{2}+2\omega_{0}\xi s\ +\omega_{0}^{2}\\
		\therefore s&=-5.216028\pm j6.954704\\
	\end{split}\label{eq:Q4_2_3}
\end{equation}

\begin{conditions}
	\omega_0 & 8.69338 \\
	\xi & 0.6 \\
\end{conditions}\\

Equation~\eqref{eq:Q4_2_3} shows the calculation for the approximate position of the dominant poles based on the values of $\omega_0$ and $\xi$.

\begin{equation}
	\begin{split}
		G_{c}\left(s\right)=\frac{k\left(s+a\right)}{s} \hspace{1cm} G_{p}\left(s\right)=\frac{0.1}{s+8}
	\end{split}\label{eq:Q4_2_4}
\end{equation}

\begin{equation}
	\begin{split}
		0&=1\ +G_{p}\left(s\right)\cdot G_{c}\left(s\right)\\
		&=s^{2}+2\omega_{0}\xi s\ +\omega_{0}^{2}\\
		&=s^{2\ }+\left(8+0.1k\right)s+0.1ka\\
	\end{split}\label{eq:Q4_2_5}
\end{equation}

The plant and general model for a PI controller are given by Eq.~\eqref{eq:Q4_2_4}. Setting the denominator of the system to zero, we re-arrange the controller to fit the general equation Eq.~\eqref{eq:Q4_2_5}.

\begin{equation}
	\begin{split}
		2\omega_{0}\xi &= 8+0.1k\\
		\therefore k &= 24.32
	\end{split}\label{eq:Q4_2_6}
\end{equation}

\begin{conditions}
    \omega_0 & 8.6934 \\
    \xi & 0.6         \\
\end{conditions}\\

\begin{equation}
	\begin{split}
		\omega_{0}^{2} &= 0.1ka\\
		\therefore a &= 31.0752
	\end{split}\label{eq:Q4_2_7}
\end{equation}

\begin{conditions}
    \omega_0 & 8.6934 \\
    k & 24.32         \\
\end{conditions}\\
Equation~\eqref{eq:Q4_2_6}-\eqref{eq:Q4_2_7} shows the calculation to find the constants of proportionality and integration $k$ and $a$ by comparing coefficients using Eq.~\eqref{eq:Q4_2_5} such that the system meets the specification set out by $\omega_0$ and $\xi$.

\begin{equation}
	\begin{split}
		G_{p}\left(s\right)&=\frac{15}{\left(s\ +\ 150\right)\left(s\ +\ 8\right)}\\
		&= \frac{15}{s^{2}+158s+1200}\\
	\end{split}\label{eq:Q4_2_8}
\end{equation}

\begin{equation}
	\begin{split}
		G_{c}\left(s\right)&=\frac{k\left(s+a\right)}{s}\\
		&=\frac{24.32\left(s+31.075\right)}{s}\\
		&=\frac{24.32s+755.7}{s}\\
	\end{split}\label{eq:Q4_2_9}
\end{equation}

\begin{conditions}
    k & 24.32  \\
    a & 31.075 \\
\end{conditions}\\

\begin{equation}
	\begin{split}
		G\left(s\right)&=G_{p}\left(s\right) \cdot G_{c}\left(s\right)\\
		&=\frac{364.8+11340}{s^{3}+158s^{2}+1200s}\\
	\end{split}\label{eq:Q4_2_10}
\end{equation}

\begin{conditions}
    G_{p}\left(s\right) & $\left(\frac{15}{s^{2}+158s+1200}\right)$ \\
    G_{c}\left(s\right) & $\left(\frac{24.32s+755.7}{s}\right)$     \\
\end{conditions}\\

\begin{equation}
	\begin{split}
		H\left(s\right)&= \frac{G_{p}\left(s\right) \cdot G_{c}\left(s\right)}{1 + G_{p}\left(s\right) \cdot G_{c}\left(s\right)}\\
		&=\frac{364.8s+11340}{s^{3}+158s^{2}+1565+11340}\\
	\end{split}\label{eq:Q4_2_11}
\end{equation}

\begin{conditions}
    G_{p}\left(s\right) & $\left(\frac{15}{s^{2}+158s+1200}\right)$ \\
    G_{c}\left(s\right) & $\left(\frac{24.32s+755.7}{s}\right)$     \\
\end{conditions}\\

Equations~\eqref{eq:Q4_2_8}-\eqref{eq:Q4_2_11} shows the plant, controller, forward path and closed loops transfer functions as designed for the motor system.

\renewcommand{\ex}{Q4_2}

\begin{figure}[ht!]
	\centering
	\input{matlab/Output/\ex.tex}
	\caption{Root Locus plot of the closed loop transfer function \eqref{eq:Q4_2_11}. \appendixamble{\ex}}
	\label{fig:\ex}
\end{figure}\FloatBarrier

Figure~\ref{fig:\ex} shows the dominant poles passing through $-5.03 \pm 7.16$ which is close to the value of $-5.216028\pm j6.954704$ calculated in Eq.~\eqref{eq:Q4_2_3}. Therefore the system will behave as designed. The calculations simplifies the third order equation to a second order equation, therefore is only an approximation and we would not expect the roots of the actual system to be exactly the same as the calculated values.


\begin{figure}[ht!]
	\centering
	\input{matlab/Output/\ex_1.tex}
	\caption{Step Response of the closed loop transfer function \eqref{eq:Q4_2_11}. \appendixamble{\ex}}
	\label{fig:\ex_1}
\end{figure}\FloatBarrier

Figure~\ref{fig:\ex_1} shows the amplitude at 750ms to be 0.993, therefore is within the specification of $2 \%$ by 750ms.
