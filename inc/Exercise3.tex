\renewcommand{\ex}{ex3}

\begin{equation}
	G_{p}\left(s\right)=\frac{10a}{\left(s+10\right)\left(s+a\right)}
	\label{eq:\ex}
\end{equation}

\begin{figure}[ht!]
	\centering
	\input{matlab/Output/\ex.tex}
	\caption{Graph showing step response of the transfer function shown in \eqref{eq:\ex} where a = 1 (blue), 10 (red), 20 (green) and 100 (black).\appendixamble{\ex}}
	\label{fig:\ex}
\end{figure}\FloatBarrier


\begin{table}[ht!]
	\centering
	\begin{tabular}{|c|c|}
		\hline
		a{[}arb{]} & Time Constant $\tau${[}s{]} \\ \hline
		1          & 1.1                         \\ \hline
		10         & 0.215                       \\ \hline
		20         & 0.159                       \\ \hline
		100        & 0.111                       \\ \hline
	\end{tabular}
    \caption{Table to show the time constant of the system \eqref{eq:\ex} for different values of a}\label{table:\ex}
\end{table}\FloatBarrier

Figure~\ref{fig:\ex}. shows the step response of the system for different values of $a$. The steady state gain of the system is $10a/10a$ which is equal to 1 for all values of $a$. The time constant of the system is shown in Table~\ref{table:\ex}. when a is very small, the pole at -a is considered dominant, therefore the time constant approaches the reciprocal of -a. When a is large, the pole at -10 is considered dominant and therefore the time constant is approaches the reciprocal of -10. \\
