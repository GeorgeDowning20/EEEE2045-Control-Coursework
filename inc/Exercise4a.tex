\renewcommand{\ex}{ex4a}

\begin{equation}
	G_{p}\left(s\right)=\frac{10}{s^2 + as + 10}
	\label{eq:\ex}
\end{equation}


\begin{figure}[ht!]
	\centering
	\input{matlab/Output/\ex.tex}
	\caption{Graph showing step response of the transfer function shown in \eqref{eq:\ex} where a = 0.2 (red), 0.6 (green), 1.2 (blue), 1.6 (black) and 3.0 (magenta). \appendixamble{\ex}}
	\label{fig:\ex}
\end{figure}\FloatBarrier


\begin{table}[ht!]
    \centering
    \begin{tabular}{|c|c|c|}
    \hline
    a{[}arb{]} & Percentage Peak Overshoot $M_p[\%]$ & damping factor $\xi[arb]$ \\ \hline
    0.2        & 91                                  & 0.0316                    \\ \hline
    0.6        & 74                                  & 0.9487                    \\ \hline
    1.2        & 54                                  & 0.1897                    \\ \hline
    1.6        & 44                                  & 0.2530                    \\ \hline
    3.0        & 18                                  & 0.4743                    \\ \hline
    \end{tabular}
    \caption{Table to show the percentage peak overshoot and damping factor of the system \eqref{eq:\ex} for different values of a}\label{table:\ex}
    \end{table}\FloatBarrier

    \begin{equation}
    \begin{split}
        \omega_0 &= \sqrt{10}\\
        \therefore \xi &= \frac{a}{2\omega_0}\\
    \end{split}\label{eq:12}
    \end{equation}

Eq.~\eqref{eq:12} shows the working to calculate the damping factor $\xi$ in table \ref{table:\ex}. The damping factor is the ratio of the damping coefficient to the natural frequency of the system. The damping coefficient is the coefficient of the second order term in the differential equation. The damping coefficient is equal to $a$ in this case. The natural frequency is the square root of the coefficient of the first order term in the differential equation. The coefficient of the first order term  $\xi$ is equal to 10 in this case. Therefore the damping factor is equal to $\frac{a}{2\sqrt{10}}$.

Figure~\ref{fig:\ex}. shows the step response of the system for different values of $a$. The steady state gain of the system is $10/10$ which is equal to 1 for all values of $a$. The percentage peak overshoot and damping factor of the system is shown in Table~\ref{table:\ex}. when a is small the percentage peak overshoot approaches 1 and the damping factor is small. When a is large, the percentage peak overshoot is small and the damping factor is large. \\

The variable a changes the roots of the denominator of the transfer function, separating the roots as a gets larger. As the ratio of imaginary to real parts of the roots increases, the damping factor increases. As the ratio of imaginary to real parts of the roots decreases, the damping factor decreases. 

\begin{equation}
    \frac{\pi}{\omega_0\sqrt{1-\xi^2}}
    \label{eq:1}
\end{equation}

While the damping factor is less than 1, the system is underdamped. This means the system will oscillate about the steady state value. The time to maximum overshoot can only be observed when the system is underdamped and is characterized by the equation \eqref{eq:1}. when the damping factor is small, the time to maximum overshoot approaches a constant value. When the damping value approaches 1, the time to maximum overshoot approaches infinity.



