\renewcommand{\ex}{ex1} 

\begin{equation}
    G_{p}\left(s\right)=\frac{a}{s+20}
    \label{eq:\ex}
\end{equation}

\begin{equation}
    \lim_{s\to0}\left(G_{p}\left(s\right)\right)
    \label{eq:\ex.1}
\end{equation}

\begin{figure}[ht!]
    \centering
    \input{matlab/Output/\ex.tex}
    \caption{Graph showing step response of plant \eqref{eq:\ex}. \appendixamble{\ex}}
    \label{fig:\ex}
\end{figure}\FloatBarrier

The plant transfer function for a simple first order system is given by \eqref{eq:\ex} where $a$ is the gain of the system and $s$ is the Laplace variable. Steady state is achieved when s approaches 0 as described by \eqref{eq:\ex.1}. Therefore the steady state gain of the system is $a/20$.

When $a=10$ the steady state gain is $\frac{10}{20}$ which is $0.5$.

Unity gain is achieved when the gain is equal to 1 at steady state. Hence the gain of the system is equal to 1 when $a=20$. The step response of such system is shown in Figure~\ref{fig:\ex}.

The time constant is the time taken for the system to reach $1-e^{-1}$ or approximately 63.2\% of its final value. The time constant is equal to the reciprocal of the denominator at steady state. Hence the time constant is equal to 0.05 at all values of $a$.




