\begin{equation}
    G_{p}\left(s\right)=\frac{a}{s+20}
    \label{eq:ex1}
\end{equation}

\begin{equation}
    \lim_{s\to0}\left(G_{p}\left(s\right)\right)
    \label{eq:ex1.1}
\end{equation}

\begin{figure}[ht!]
    \centering
    % This file was created by matlab2tikz.
%
%The latest updates can be retrieved from
%  http://www.mathworks.com/matlabcentral/fileexchange/22022-matlab2tikz-matlab2tikz
%where you can also make suggestions and rate matlab2tikz.
%
\definecolor{mycolor1}{rgb}{0.00000,0.44700,0.74100}%
%
\begin{tikzpicture}

\begin{axis}[%
width=4.442in,
height=3.393in,
at={(0.893in,0.486in)},
scale only axis,
separate axis lines,
every outer x axis line/.append style={white!40!black},
every x tick label/.append style={font=\color{white!40!black}},
every x tick/.append style={white!40!black},
xmin=0,
xmax=0.45,
every outer y axis line/.append style={white!40!black},
every y tick label/.append style={font=\color{white!40!black}},
every y tick/.append style={white!40!black},
ymin=0,
ymax=1,
axis background/.style={fill=white}
]
\addplot [color=mycolor1, forget plot]
  table[row sep=crcr]{%
0	0\\
0.00460517018598805	0.0879891606440893\\
0.0092103403719761	0.168236228897327\\
0.0138155105579642	0.241422424970814\\
0.0184206807439522	0.308169029081061\\
0.0230258509299403	0.369042655519804\\
0.0276310211159283	0.42456006266284\\
0.0322361913019162	0.475192539750224\\
0.0368413614879043	0.521369907677358\\
0.0414465316738923	0.56348416775983\\
0.0460517018598804	0.601892829446499\\
0.0506568720458684	0.636921945229895\\
0.0552620422318565	0.668868878517405\\
0.0598672124178445	0.698004827959794\\
0.0644723826038326	0.724577129666179\\
0.0690775527898206	0.748811356849038\\
0.0736827229758087	0.770913234723219\\
0.0782878931617967	0.791070386914592\\
0.0828930633477848	0.809453928203672\\
0.0874982335337728	0.826219917125059\\
0.0921034037197609	0.841510680753885\\
0.0967085739057488	0.855456022925404\\
0.101313744091737	0.868174326144356\\
0.105918914277725	0.879773556538256\\
0.110524084463713	0.890352180385679\\
0.115129254649701	0.899999999999998\\
0.119734424835689	0.908798916064407\\
0.124339595021677	0.916823622889731\\
0.128944765207665	0.92414224249708\\
0.133549935393653	0.930816902908105\\
0.138155105579641	0.936904265551979\\
0.142760275765629	0.942456006266283\\
0.147365445951617	0.947519253975021\\
0.151970616137605	0.952136990767735\\
0.156575786323593	0.956348416775982\\
0.161180956509581	0.960189282944649\\
0.165786126695569	0.963692194522989\\
0.170391296881558	0.96688688785174\\
0.174996467067546	0.969800482795979\\
0.179601637253534	0.972457712966617\\
0.184206807439522	0.974881135684903\\
0.18881197762551	0.977091323472322\\
0.195719732904492	0.98004737685031\\
0.202627488183474	0.982621991712506\\
0.209535243462456	0.984864387515637\\
0.216442998741438	0.986817432614435\\
0.22335075402042	0.988518463785031\\
0.230258509299402	0.99\\
0.239468849671378	0.991682362288973\\
0.248679190043354	0.993081690290811\\
0.25788953041533	0.994245600626628\\
0.2694024558803	0.995429118103851\\
0.28091538134527	0.996369219452299\\
0.294730891903235	0.997245771296662\\
0.310848987554193	0.998004737685031\\
0.329269668298145	0.998619615735397\\
0.352295519228085	0.999129036410044\\
0.382229125437007	0.999521369907678\\
0.4236756571109	0.999791070386915\\
0.451306678226828	0.999879773556539\\
};
\addplot [color=black, dotted, forget plot]
  table[row sep=crcr]{%
0	1\\
0.495	1\\
};
\end{axis}

\begin{axis}[%
width=4.568in,
height=3.603in,
at={(0.766in,0.486in)},
scale only axis,
xmin=0,
xmax=1,
xtick={\empty},
xlabel={Time (seconds)},
ymin=0,
ymax=1,
ytick={\empty},
ylabel={Amplitude},
axis line style={draw=none},
ticks=none,
title style={font=\bfseries},
title={Step Response},
axis x line*=bottom,
axis y line*=left
]
\end{axis}
\end{tikzpicture}%
    \caption{Graph showing step response of plant }
    \label{fig:ex1}
\end{figure}\FloatBarrier

The plant transfer function for a simple first order system is given by \eqref{eq:ex1} where $a$ is the gain of the system and $s$ is the Laplace variable. Steady state is achieved when s approaches 0 as described by \eqref{eq:ex1.1}. Therefore the steady state gain of the system is $a/20$.

When $a=10$ the steady state gain is $\frac{10}{20}$ which is $0.5$.

Unity gain is achieved when the gain is equal to 1 at steady state. Hence the gain of the system is equal to 1 when $a=20$. The step response of such system is shown in Figure~\ref{fig:ex1}.

The time constant is the time taken for the system to reach $1-e^{-1}$ or approximately 63.2\% of its final value. The time constant is equal to the reciprocal of the denominator at steady state. Hence the time constant is equal to 0.05 at all values of $a$.




